\chapter*{Zusammenfassung}
%Die deutsche Zusammenfassung. Halbe Seite.

% An der Universität Greifswald werden zur Unterstützung der Lehrenden und Lernenden ``\acs{Moodle}'' 
% und “\acs{LSF}” eingesetzt. Bei \acs{Moodle} handelt es sich dabei um ein \ac{Course-Management-System}, 
% welches kooperative Lehr- und Lernmethoden bereitstellt. Mit der
% Hilfe dieses Systems werden online Kursräume bereitgestellt, in denen Arbeitsmaterialien 
% und Lernaktivitäten veröffentlicht werden.
% Neben diesem System existiert an der Universität Greifswald ein Modul des Hochschul-Informationssystems 
% “\acs{HIS}” mit dem Namen “\acs{LSF}”, welches zur Repräsentation von
% Lehrveranstaltungen und allen dazu benötigten Ressourcen (wie z.B. Räume, Personen
% oder Lehrmittel) genutzt wird.
% Es soll nun im Rahmen einer Bachelorarbeit eine Schnittstelle zwischen beiden Systemen 
% erstellt werden, welche eine Kommunikation zwischen diesen ermöglicht. Es soll
% damit erreicht werden, dass die in \acs{HIS}-\acs{LSF} enthaltenen Daten auch innerhalb von \acs{Moodle}
% nutzbar sind, ohne die Inhalte per Hand zu übertragen und damit den Aufwand für Aktualisierungen zu verringern. 

Innerhalb der Gebäude der Fachhochschule Stralsund werden vor den Räumen Türschilder, Schaukästen oder Pinnwände genutzt, um allgemeine und raumspezifische Ankündigungen, mit Hilfe von Aushängen, zu verteilen. Damit ist ein einfacher Austausch von Informationen zwischen den Institutionen der Fachhochschule und deren Angehörigen möglich. 
Im Rahmen einer Projektarbeit soll nun eine Möglichkeit geschaffen werden, diese Aushänge in eine elektronische Form zu bringen, um so zum einen den Verwaltungsaufwand zu verringern und eine Möglichkeit zu schaffen Informationen effektiver zu verteilen. 