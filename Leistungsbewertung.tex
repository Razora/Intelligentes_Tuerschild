\chapter{ Ausblick }\label{chap:Ausblick}
\vspace*{-3cm}
\begin{flushleft}
In diesem Kapitel wird ein kurzer Überblick gegeben, welche Erweiterungen noch für das Projekt möglich wären und erste Ansätze wie diese umgesetzt werden könnten. 

\section{Wireless-ADB}
Im bisherigen Verlauf des Projektes werden viele Einstellungen (wie z.B. das Rooten oder die Software-Installation) über ADB vorgenommen. Nachteil an dieser Methode ist, dass das Tablet jedes mal per USB-Kabel an einen PC angeschlossen werden muss. Dies bedeutet, dementsprechend, dass diese von der Wand abmontiert werden müssen, um neue Einstellungen vorzunehmen. Eine Lösung dafür wäre, dass die ADB-Befehle nicht mit Hilfe eines USB-Kabels übertragen werden, sondern mit Hilfe einer WLan-Verbindung. Dies wird jedoch standardmäßig nicht vom Odys Xelio unterstützt. Abhilfe verschaffen jedoch Apps, welche diese Funktion nachrüsten. Diese sind im ``Google Play Store'' (bezeichnet eine Plattform, welche das herunterladen und installieren von Apps auf Android Geräten ermöglicht) unter dem Begriff ``Wireless ADB'' zu finden. Somit ist der Nutzer nicht mehr dazu gezwungen ein USB-Kabel für die Konfiguration zu nutzen, sondern kann alle Einstellungen über Wlan vornehmen, solange sich das Tablet und der Router in einem Netzwerk aufhalten und die entsprechende App aktiviert ist. Somit ist nun auch nicht mehr nötig das Tablet für Einstellungen von der Wand abzunehmen. 
Jedoch ist es nun nicht einfach möglich die zum Projekt mitgegebenen .batch- bzw. .bash-Dateien zu nutzen, denn diese geben keinen Aufschluss darüber, wo sich das Gerät im Netzwerk befindet. Eine Möglichkeit dieses Missstand zu beheben, wäre es für jedes zu verwaltende Tablet einen angepassten Satz .batch- bzw. .bash-Dateien anzulegen. Da dies jedoch den Verwaltungsaufwand stark erhöhen würde, muss eine andere Lösung her. Ein Ansatz wäre somit die dynamische Erzeugung der Dateien. Dafür muss das beigelegte Verwaltungsprogramm, welches unter anderem für das Rooten oder das aktivieren bzw. deaktivieren der Hardwarebuttons verantwortlich ist, erweitert werden. Bei dieser Erweiterung handelt ist sich zum einen um eine Möglichkeit eine gültige IP-Adresse einzugeben (dieses Feld sollte mit einem Radio-Button aktiviert werden, um eine bessere Unterscheidung zwischen den beiden Methoden zu ermöglichen) und zum anderen die dynamische Erstellung von entsprechenden Dateien. Da bei der WLan-Verbindung zu einem Tablet eine jedes mal eine andere IP-Adresse vorhanden sein kann, muss für jede Aktion eine neue Datei geschrieben und daraufhin ausgeführt werden. Wie das Schreiben einer Datei aussehen könnte, zeigt folgendes Listing (gezeigt am Beispiel des Rootens):
\newpage
\begin{lstlisting}[caption={Dynamische Erstellung einer .batch-Datei}\label{dynamische Dateierstellung},captionpos=t,linewidth=\textwidth,language=JAVA] 
 
//erstellen der zu schreibenden .bat-Datei
File file = new File(System.getProperty("user.dir") + "\\ADB\\wireless_mkroot.bat");
try{
//Datei wird immer neu ueberschrieben
FileWriter writer = new FileWriter(file);	

//es wird abgefragt, ob ein Radio-Button mit Namen "rdbtnOnline" aktiviert ist
//falls ja: die geforderte Einstellung wird mit Wireless-ADB ausgefuehrt
//falls nein: die geforderte Einstellung wird mit kabelgebundenen ADB ausgefuehrt
if(rdbtnOnline.isSelected() == true)
{
	//es wird abgefragt, ob das Feld fuer die IP ausgefuellt wurde
	if(IPFeld.getIP() != null)
	{
		//es wird der noetige string in einen zuvor erstellten FileWriter geschrieben
		//dabei unterscheidet sich diese Anweisungen in zwei Punkten
		//von der USB-Variante:
		//erster Punkt:
		//"adb disconnect " + IPFeld.getIP() + "\n" +
		//dieser Befehl fuehrt zur Sicherheit eine Trennung des Geraetes 
		//mit der ausgelesenen IP durch
		//zweiter Punkt:
		//"adb connect "  + IPFeld.getIP() + "\n" +
		//dieser Befehl sorgt nun dafuer, dass eine Wlan-Verbindung zum Tablet
		//hergestellt wird und ADB-Befehle gesendet werden koennen 
		//alle weiteren Kommandos entsprechen der USB-Variante
		writer.write(
				"@echo off\n" +
				"setlocal\n" +
				"set version=0.3\n" +
				"title %~nx0 - version %version% - by GK\n" +
				"mode CON: COLS=100 LINES=44\n" +
				"color 1E\n\n" +
				
				"cd ADB\n\n" +
				
				"if not exist su (\n" +
				"echo ERROR: file 'su' not found in %~dp0 !\n" +
				"goto END\n" +
				")\n\n" +

				"if not exist Superuser.apk (\n" +
				"echo ERROR: file 'Superuser.apk' not %~dp0 !\n" +
				"goto END\n" +
				")\n\n" +
				  
				"adb disconnect " + IPFeld.getIP() + "\n" +
				
				"adb connect "  + IPFeld.getIP() + "\n" +
				
				"echo Waiting for device ...\n" +
				"adb wait-for-device\n" +
				"echo Restarting ADB with root permission ...\n" +
				"adb root\n" +
				"echo Remounting /system RW ...\n" +
				"adb remount\n" +
				"echo Copying su to /system/bin/ ...\n" +
				"adb push su /system/bin/\n" +
				"echo Copying Superuser.apk to /system/app/ ...\n" +
				"adb push Superuser.apk /system/app/\n" +
				"echo Setting permission for /system/bin/su\n" +
				"adb shell /system/bin/busybox chmod 6755 /system/bin/su\n" +
				"REM echo Setting permission for /system/app/Superuser.apk\n" +
				"REM adb shell /system/bin/busybox chmod 6755 /system/app/Superuser.apk\n" +
				"echo Removing existing /system/xbin/su\n" +
				"adb shell /system/bin/busybox rm -f /system/xbin/su\n" +
				"echo Creating symlink to /system/bin/su in /system/xbin\n" +
				"adb shell /system/bin/busybox ln -s /system/bin/su /system/xbin/su\n" +
				"echo Now rebooting device ...\n" +
				"adb reboot\n" +
				":END\n" +
				"echo.\n" +
				"pause\n" +
				"exit");
				
		//Stream in Datei schreiben
		writer.flush();
								
		//Stream schliessen
		writer.close();
								
		Runtime.getRuntime().exec("cmd /C start " + System.getProperty("user.dir") + 
		"\\ADB\\wireless_mkroot.bat");
	}
	else{
		JOptionPane.showMessageDialog(null, "Bitte geben Sie eine gueltige IP-Adresse an", 
		"Fehler", JOptionPane.OK_OPTION);
	}
}
else
{
	Runtime.getRuntime().exec("cmd /C start " + System.getProperty("user.dir") + 
	"\\ADB\\mkroot.bat"); 
}	
	}
		catch (IOException e) {
			// TODO Auto-generated catch block
			e.printStackTrace();
		}
	}
}
\end{lstlisting}

Auf dem im Listing gezeigten Wege ist also nun möglich für jedes über Wlan angeschlossene Gerät eine entsprechende Datei zu erzeugen, die dann die angegebenen Befehle an das Gerät sendet. Somit ist es nun leichter möglich die Konfiguration des Odys Xelio zu verändern. Diese Vorgehensweise kann für jede andere Einstellung (Softwareinstallation oder die Aktivierung der Hardwarebuttons) genutzt werden. 

Jedoch hat diese Technik nicht den Weg in die Ausarbeitung des Projektes gefunden, da die gegebenen WLan-ADB-Apps nicht die Möglichkeit bieten die Verbindung mit einem Passwort zu sichern. Somit wäre es für jede Person möglich, welche die IP-Adresse des Gerätes besitzt, Kontrolle über das Tablet zu erlangen und Veränderungen daran vorzunehmen. Um dies zu verhindern muss also die eingesetzte WLan-ADB-App die Option bieten vor der Verbindung ein Passwort abzufragen. Eine weiteres Mittel um Wireless-ADB auch mit den bisher existierenden Wireless-ADB-Apps einzusetzen, wäre die Einrichtung eines eigenen Netzwerkes für die Tablets innerhalb des Gebietes in dem sie eingesetzt werden. So könnte die Anzahl an Personen, welche Zugriff auf die Tablets haben, stark begrenzt werden und es könnte somit auch vorher eine genaue Auswahl getroffen werden, welche Personen die Rechte haben Einstellungen zu verändern.

\section{Konzept zur Netzwerkanbindung}
Um die intelligenten Türschilder in der gesamten Fachhochschule bzw. erstmal im Fachbereich ETI zu realisieren, muss noch ein Lösung erarbeitet werden wie man die Türschilder an ein Netzwerk mit Samba Freigaben anbindet.\\

An dieser Stelle möchte wir ein mögliches Konzept dafür vorstellen. Eine erste Idee wäre es die Türschilder zum Beispiel in ein Campus weites bestehendes WLan mit einzubinden und dort Samba Shares zu erlauben. Da dann womöglich sehr viele Leute im gleichen Netzwerk sind, müsste man sich bei den späteren Freigaben für die Türschilder ziemlich sichere Passwörter verwenden, um Missbrauch vorzubeugen. Zu sichere Passwörter würden nach unserer Meinung aber der Komfortabilität entgegenwirken. Daher wäre es besser ein eigenes WLan einzurichten, was wiederum auch für Wireless-ADB von Vorteil wäre.\\

In diesem WLan könnte dann eine Virtuelle Maschine (VM) mit einem Samba Dienst eingehängt sein. Solch ein VM wurde von uns auch vorbereitet und wird auf einem Datenträger als OVA-(Open Virtualization Format)-Archiv mitgeliefert.\\

Diese vorbereitete VM enthält ein Debian Squeeze System mit installierten Samba Dienst. Der Dienst ist so konfiguriert das man einen Nutzer auf dem System anlegt z.B. als Nutzername ``raum319'', dann wird automatisch das Home-Verzeichnis des Nutzers, was den selben Namen als Verzeichnisnamen hat freigegeben.\\

Um einen korrekten Nutzer für das Linux System und den Samba Dienst zu erzeugen, wurde außerdem ein Skript erstellt, was unter dem Home-Verzeichnis des root-Nutzers liegt.\\

Dieses ruft man dann wie folgt auf:
\vspace{.3cm}
\begin{center}
\verb|./create_idc_samba_user %Nutzername% %Passwort%|
\end{center}
\vspace{.3cm}
Mithilfe einer zeilenweisen Liste von Nutzernamen und Passwörtern, könnte man dann auch relativ schnell mehrere Nutzer anlegen.\\

Daraufhin kann man die Synchronisation Applikation auf den Tablets wie im Abschnitt \ref{pcfilesync_subsec} \nameref{pcfilesync_subsec} konfigurieren.\\

Außerdem muss die VM von den PCs der Professoren bzw. von allen Leuten erreichbar sein, die ein solches Türschild mit neuen Daten bestücken wollen. Dazu muss dann nur die entsprechende Freigabe auf dem Rechner geöffnet werden und dort die Bilddateien gelöscht bzw. hinzugefügt werden. 

% Ich denke hier sollte wir/ich noch mal meine Samba VM erwähnen, wie ich mir dem Umgang mit dieser vorgestellt habe

\end{flushleft}
% Zu jeder Arbeit in unserem Bereich gehört eine Leistungsbewertung. Aus diesem Kapitel sollte hervorgehen, welche Methoden angewandt worden, die Leistungsfähigkeit zu bewerten und welche Ergebnisse dabei erzielt wurden. Wichtig ist es, dem Leser nicht nur ein paar Zahlen hinzustellen, sondern auch eine Diskussion der Ergebnisse vorzunehmen. Sehr gut ist, wenn man zunächst diskutiert und plausibel macht, welche Ergebnisse man erwartet, und dann eventuelle Abweichungen diskutiert.
