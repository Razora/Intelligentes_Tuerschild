
% Diesen Part des Dokuments als letztes schreiben


\chapter{ Einleitung }
\vspace{-3cm}
\begin{flushleft}
% 
% Die Einleitung schreibt man zuletzt, wenn die Arbeit im Großen und Ganzen schon fertig ist.
% (Wenn man mit der Einleitung beginnt - ein häufiger Fehler - braucht man viel länger und wirft sie später doch wieder weg). 
% Sie hat als wesentliche Aufgabe, den Kontext für die unterschiedlichen Klassen von Lesern herzustellen. Man muß hier die Leser für sich gewinnen. 
% Das Problem, mit dem sich die Arbeit befasst, sollte am Ende wenigsten in Grundzügen klar sein und dem Leser interessant erscheinen. 
% Das Kapitel schließt mit einer Übersicht über den Rest der Arbeit. Meist braucht man mindestens 4 Seiten dafür, mehr als 10 Seiten liest keiner.
% 
% Diese fragen Sollten beantwortet werden:
% Welche Fragestellung will ich behandeln ?
% Warum ist sie wichtig ?
% Warum ist sie nicht trivial ? ( bereits gelöst oder einfach lösbar )
% was möchte ich neues zur Lösung der Fragestellung beitragen ?
% Welche empirische Behandlung wähle ich zur Lösung der Fragestellung ?

\section{Motivation}
\label{sec:Motivation}
Im Rahmen einer Projektarbeit an der Fachhochschule Stralsund soll eine Methode entwickelt werden, um die innerhalb der Fachhochschulgebäude 
an den Türen angebrachten Schildern durch eine elektrische Lösung zu ersetzen. Dabei besteht besteht die Hauptaufgabe darin den Schildern einen größeren Funktionsumfang zu geben und die dafür notwendige Anzahl an Arbeitsschritten so gering wie mögliche zu halten.  

\section{Bedeutung der Arbeit}
\label{sec:Bedeutung der Arbeit}
Die bisher an den Türen angebrachten Schildern werden hauptsächlich dafür genutzt um die Belegungspläne des jeweiligen Raumes oder um wichtige Informationen von Professoren zu veröffentlichen. Dabei ist darauf zu achten, dass nur ein begrenzt nutzbarer Raum für Informationen bereitsteht. Dadurch kommt es oft dazu, dass keine neuen Bekanntmachungen an den Schildern angebracht werden können und diese so verloren gehen, da sie nur in einem eingeschränkten Zeitraum aktuell sind. Dieser Mangel im Informationsfluss kann somit auch dazu führen, dass den Studenten Möglichkeiten entgehen, die ihre Studienzeit oder auch die erste Zeit im Berufsleben beeinflussen könnten. Beispiele dafür währen die Mitteilungen von Studien- oder Abschlussarbeiten an der Fachhochschule selbst oder in der freien Wirtschaft. 
Als nächster Punkt kann die Betreuung mehrerer Räume aufgeführt werden. Es muss in der derzeitigen Situation jeder Raum einzeln betreut werden, was eine vermehrte Zahl von Arbeitsschritten verlangt. Wird allerdings eine zentrale Lösung zur Verwaltung geschaffen, kann so die Zahl der nötigen Verwaltungsvorgänge verringert werden. 

\section{Zielsetzung}
\label{sec:Zielsetzung}
Mit dieser Arbeit soll eine Möglichkeit geschaffen werden, um die oben genannten Missstände zu beheben. Diese besteht darin die derzeitigen Türschilder durch eine elektrische Alternative zu ersetzen. Diese muss daraufhin so konfiguriert werden, dass sie eine Möglichkeit bietet zentral verwaltet zu werden und muss jedoch gleichzeitig gegen unbefugte Eingaben von Dritten geschützt werden. 

\end{flushleft}
