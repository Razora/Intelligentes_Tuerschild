
% Diesen Part des Dokuments als letztes schreiben


\chapter{ Einleitung }

Die Einleitung schreibt man zuletzt, wenn die Arbeit im Großen und Ganzen schon fertig ist.
(Wenn man mit der Einleitung beginnt - ein häufiger Fehler - braucht man viel länger und wirft sie später doch wieder weg). 
Sie hat als wesentliche Aufgabe, den Kontext für die unterschiedlichen Klassen von Lesern herzustellen. Man muß hier die Leser für sich gewinnen. 
Das Problem, mit dem sich die Arbeit befasst, sollte am Ende wenigsten in Grundzügen klar sein und dem Leser interessant erscheinen. 
Das Kapitel schließt mit einer Übersicht über den Rest der Arbeit. Meist braucht man mindestens 4 Seiten dafür, mehr als 10 Seiten liest keiner.

Diese fragen Sollten beantwortet werden:
Welche Fragestellung will ich behandeln ?
Warum ist sie wichtig ?
Warum ist sie nicht trivial ? ( bereits gelöst oder einfach lösbar )
was möchte ich neues zur Lösung der Fragestellung beitragen ?
Welche empirische Behandlung wähle ich zur Lösung der Fragestellung ?

