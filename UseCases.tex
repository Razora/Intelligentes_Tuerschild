\subsection{User Stories}
\label{sec:Usecases}
\begin{flushleft}
% eine einleitung wenn hier viel text folgt.
In diesem Abschnitt soll die geplante Verwendung der Soft- und Hardware erklärt werden.
Dafür werden die verschiedenen User Stories vorgestellt.

% warum kommt es zum einsatz ?
% Eine Softwarelösung für das Problem der Zahnbewegung soll hauptsächlich für zwei Dinge erfolgen.
% Zum einen soll die Software für die Lehre benutzt werden und zum anderen um Voraussagen für konkrete Behandlungen zu treffen.

% Die ``Use cases'' für diese Softwareentwicklung sind im Appendix Abschnitt \ref{sec:Use cases} auf Seite \pageref{sec:Use cases} zu finden, 
% werden hier aus Übersichtlichkeisgründen aber kurz erklärt.

\subsubsection{Dokumente auf das Türschild übertragen} % (fold)
\label{ssub:Daten auf das Türschild übertragen}
Als User möchte ich Dokumente auf das Türschild übertragen, um diese zu publizieren. 

\subsubsection{Dokumente vom Türschild entfernen}
\label{ssub:Dokumente vom Speicher des Türschilds entfernen}
Als User möchte ich Dokumente vom Türschild löschen, um veraltete Informationen nicht mehr zu verbreiten.  

\subsubsection{Das Türschild gegen Eingaben von Dritten schützen}
\label{ssub:Das Türschild gegen Eingaben von Dritten schützen}
Als Türschildverwalter möchte ich das Türschild so einstellen, dass fremde Personen keine Veränderungen daran vornehmen können. 
\newline
\end{flushleft}



% Der Benutzer der Software soll in der Lage sein präparierte Inventor-Daten zu Laden.
% %% TODO erklärung des dateiformats ?
% Der Aufbau des Dateiformats wird im Anhang Abschnitte \ref{sec:Dateiformat} erklärt.
% Hierbei soll es dem Nutzer möglich sein die Datei durch einen Dateiauswahldialog auszuwählen um sie dann in das Programm zu laden.
% Sind die Daten geladen zeigt sich dem Nutzer eine Visuelle repräsentation der Zähne.



