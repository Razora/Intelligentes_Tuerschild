
%
% Vorgehen : 
%           Darauf los schreiben
%           Ideen sammeln
%           Ordnen
%           Struktorieren
%           Einleitung
%           Überschrift suchen
%
%Formatänderung: in documentclass ``openany'' eingefügt. Inhaltsverzeichnis in der .include/Hyperref ``linktocpage'' eingefügt
\documentclass[bibtotoc,liststotoc,BCOR5mm,DIV12,openany]{scrbook}
\usepackage{babelbib}       % deutsche Literaturverzeichnisse
\usepackage[utf8]{inputenc}
\usepackage[ngerman]{babel}
\usepackage{graphicx} % Einfügen von Grafiken  - für PDF-Latex: .pdf und .png (.jpg möglich, sollte aber vermieden werden)
\usepackage{url}           % URL's (z.B. in Literatur) schöner formatieren
\usepackage[T1]{fontenc}
\usepackage{array}
\usepackage{listings}
\usepackage{color}
\usepackage[none]{hyphenat}


%% Graphiken hier rein
\graphicspath{{./Bilder/}}

% ------------------------------------------------------------------------------
% \input{hyphenation} % in dieses File kommen Wörter die Latex nicht richtig trennt

%--------------------- globale includes 

% hyperref 
\usepackage{hyperref} % sorgt für für Hyperlinks in PDF-Dokumenten

\hypersetup{
    bookmarks=true,         % show bookmarks bar?
    unicode=false,          % non-Latin characters in Acrobat’s bookmarks
    pdftoolbar=true,        % show Acrobat’s toolbar?
    pdfmenubar=true,        % show Acrobat’s menu?
    pdffitwindow=false,     % window fit to page when opened
    pdfstartview={FitH},    % fits the width of the page to the window
    pdftitle={Projekarbeit Türschild},    % title
    pdfauthor={René Sodemann, Stefan Taute},     % author
    pdfsubject={Projektarbeit "Intelligentes Türschild"},   % subject of the document
    pdfcreator={René Sodemann, Stefan Taute},   % creator of the document
    pdfproducer={René Sodemann, Stefan Taute}, % producer of the document
    pdfkeywords={Android} {Türschild} {intelligent} {idc} {doorplate} {Samba} {Odys} {Xelio}, % list of keywords
    pdfnewwindow=true,      % links in new window
    colorlinks=true,       % false: boxed links; true: colored links
    linkcolor=blue,          % color of internal links
    citecolor=green,        % color of links to bibliography
    filecolor=magenta,      % color of file links
    urlcolor=cyan,           % color of external links
    linktocpage
}



%stefan auskommentiert%% globale glossardatei 

% glossar package initialisierung 
\usepackage[
nonumberlist, %keine Seitenzahlen anzeigen
acronym,      %ein Abkürzungsverzeichnis erstellen
toc,          %Einträge im Inhaltsverzeichnis
section]      %im Inhaltsverzeichnis auf section-Ebene erscheinen
{glossaries}

% entfernt den punkt nach der glossar beschreibung
\renewcommand*{\glspostdescription}{}

\newglossary[slg]{symbolslist}{syi}{syg}{Symbolverzeichnis}

\makeglossaries

%% NOTE: wenn die einträge nicht referenziert werden gibt es auch keinen eintrag im glossar

\newglossaryentry{symb:Pi}{
    name=$\pi$,
    description={Die Kreiszahl.},
    sort=symbolpi, type=symbolslist
}
\newacronym{MS}{MS}{Microsoft}

%\newglossaryentry{symb:Pi}{
%name=$\pi$,
%description={Die Kreiszahl.},
%sort=symbolpi, type=symbolslist
%}
% oder 

% \newacronym{MS}{MS}{Microsoft}
% \newacronym{AD}{AD}{Active Directory\protect\glsadd{glos:AD}}
% \gls{AD}




\makeatletter
\def\thickhrulefill{\leavevmode \leaders \hrule height 1pt\hfill \kern \z@}
\renewcommand{\maketitle}{\begin{titlepage}%
    \let\footnotesize\small
    \let\footnoterule\relax
    \parindent \z@
    \reset@font
    \null
    \vskip 50\p@
    \begin{center}
      \hrule
      \vskip 1pt 
      \hrule
      \vskip 1pt
      {\huge \bfseries \strut \@title \strut}\par
      \vskip 1pt
      \hrule
      \vskip 1pt
      \hrule
    \end{center}
    \vskip 50\p@
    \begin{flushright}
      \Large \@author \par
    \end{flushright}
    \vfil
    \null
  \end{titlepage}%
  \setcounter{footnote}{0}%
}

\makeatletter
\def\section{\@ifstar\unnumberedsection\numberedsection}
\def\numberedsection{\@ifnextchar[%]
  \numberedsectionwithtwoarguments\numberedsectionwithoneargument}
\def\unnumberedsection{\@ifnextchar[%]
  \unnumberedsectionwithtwoarguments\unnumberedsectionwithoneargument}
\def\numberedsectionwithoneargument#1{\numberedsectionwithtwoarguments[#1]{#1}}
\def\unnumberedsectionwithoneargument#1{\unnumberedsectionwithtwoarguments[#1]{#1}}
\def\numberedsectionwithtwoarguments[#1]#2{%
  \ifhmode\par\fi
  \removelastskip
  \vskip 3ex\goodbreak
  \refstepcounter{section}%
  \begingroup
  \noindent
  \leavevmode\Large\bfseries\raggedright
  \thesection\ #2\par\nobreak
  \endgroup
  \noindent\hrulefill\nobreak
  \vskip 2ex\nobreak
  \addcontentsline{toc}{section}{%
    \protect\numberline{\thesection}%
    #1}%
  }
\def\unnumberedsectionwithtwoarguments[#1]#2{%
  \ifhmode\par\fi
  \removelastskip
  \vskip 3ex\goodbreak
%  \refstepcounter{section}%
  \begingroup
  \noindent
  \leavevmode\Large\bfseries\raggedright
%  \thesection\ 
  #2\par\nobreak
  \endgroup
  \noindent\hrulefill\nobreak
  \vskip 2ex\nobreak
  \addcontentsline{toc}{section}{%
%    \protect\numberline{\thesection}%
    #1}%
  }

\makeatletter
\def\thickhrulefill{\leavevmode \leaders \hrule height 1ex \hfill \kern \z@}
\def\@makechapterhead#1{%
  \vspace*{10\p@}%
  {\parindent \z@ 
    {\raggedleft \reset@font
      \scshape \@chapapp{} \thechapter\par\nobreak}%
    \par\nobreak
    \vspace*{30\p@}
    \interlinepenalty\@M
    {\raggedright \Huge \bfseries #1}%
    \par\nobreak
    \hrulefill
    \par\nobreak
    \vskip 100\p@
  }}
\def\@makeschapterhead#1{%
  \vspace*{10\p@}%
  {\parindent \z@ 
    {\raggedleft \reset@font
      \scshape \vphantom{\@chapapp{} \thechapter}\par\nobreak}%
    \par\nobreak
    \vspace*{30\p@}
    \interlinepenalty\@M
    {\raggedright \Huge \bfseries #1}%
    \par\nobreak
    \hrulefill
    \par\nobreak
    \vskip 100\p@
  }}

\usepackage{amssymb,amsmath}
\usepackage{graphicx}
\usepackage{color}
\usepackage{listings}
\usepackage{courier}
\definecolor{darkgreen}{rgb}{0.0,0.6,0.0}
\lstset{
         basicstyle=\footnotesize\ttfamily, % Standardschrift
         numbers=left,               % Ort der Zeilennummern
         numberstyle=\tiny,          % Stil der Zeilennummern
         %stepnumber=2,               % Abstand zwischen den Zeilennummern
         numbersep=5pt,              % Abstand der Nummern zum Text
         tabsize=2,                  % Groesse von Tabs
         extendedchars=true,         %
         breaklines=true,            % Zeilen werden Umgebrochen
         keywordstyle=\color{darkgreen},
         frame=b,         
 %        keywordstyle=[1]\textbf,    % Stil der Keywords
 %        keywordstyle=[2]\textbf,    %
 %        keywordstyle=[3]\textbf,    %
 %        keywordstyle=[4]\textbf,   \sqrt{\sqrt{}} %
         stringstyle=\color{red}\ttfamily, % Farbe der String
         showspaces=false,           % Leerzeichen anzeigen ?
         showtabs=false,             % Tabs anzeigen ?
         xleftmargin=17pt,
         framexleftmargin=17pt,
         framexrightmargin=5pt,
         framexbottommargin=4pt,
         %backgroundcolor=\color{lightgray},
         showstringspaces=false      % Leerzeichen in Strings anzeigen ?        
 }
 \lstloadlanguages{% Check Dokumentation for further languages ...
         %[Visual]Basic
         %Pascal
         %C
         C++
         %XML
         %HTML
         %Java
 }
%\captionsetup[lstlisting]{singlelinecheck=false, labelfont={blue}, textfont={blue}}
\usepackage{caption}
\DeclareCaptionFont{blue}{\color{blue}} 
\DeclareCaptionFont{white}{\color{white}}
\DeclareCaptionFormat{listing}{\colorbox[cmyk]{0.43, 0.35, 0.35,0.01}{\parbox{\textwidth}{\hspace{15pt}#1#2#3}}}
\captionsetup[lstlisting]{format=listing,labelfont=white,textfont=white, singlelinecheck=false, margin=0pt, font={bf,footnotesize}}




%--------------------- lokale includes 
%stefan auskommentiert%
%\newglossaryentry{symb:<symbolname>}{
%    name=$<command oder zeichen>$,
%    description=<beschreibung>,
%    sort=symbolpi, type=symbolslist
%}

%\newacronym{MS}{MS}{Microsoft}



\title{Projektarbeit}
\author{René Sodemann und Stefan Taute}
\date{today\setcounter{page}{0}\clearpage}

% --------------------------------------------------------------- BEGIN
\begin{document}

\raggedright
\frontmatter % Titelblätter und Erklärung jeweils spezifisch für die jeweilige Uni einbinden
    \begin{figure}[htb]
  \centering
  \includegraphics[width=.4\textwidth]{ETI_logo.jpg}\\ % PNG-File
\end{figure}

  \centering
Projektarbeit
\vspace{\baselineskip}
im Studiengang Informatik

Fachbereich Elektrotechnik und Informatik 

\vspace{\baselineskip}
Thema: ``Intelligentes Türschild''
%% hier themanamen


\begin{center}
vorgelegt von

\vspace{\baselineskip}

Name: Sodemann \quad Vorname: René

Name: Taute \quad Vorname: Stefan

\vspace{\baselineskip}

Gutachter: Prof. Dr. rer. nat. Christian Bunse

\end{center}
    \tableofcontents     
    % Hier schreiben wir wer was gemacht hat

\chapter*{Aufgabenverteilung}

\begin{center}
\begin{tabular}[t]{|>{\centering\arraybackslash}p{.3\textwidth}|>{\centering\arraybackslash}p{.3\textwidth}|>{\centering\arraybackslash}p{.3\textwidth}|}
\hline
Aufgabe & René Sodemann & Stefan Taute\\
\hline
Versuche einen Digitalen Bilderrahmen als Intelligentes Türschild zu verwenden & X & X\\
\hline
Erstellung der Software für die Verwaltung am PC und der dazugehörigen Skripte & Software + BATCH-Skripte & BASH-Skripte\\
\hline
Erstellung einer APP für die Verwaltung auf dem Tablet &  & X\\
\hline
Recherche nach brauchbaren unterstützenden APP's & X & X\\
\hline
Ausarbeitung eines Konzepts für die Netzwerkeinbindung des Türschilds &  & Konzept (inkl. vorgefertigter VM)\\
\hline
Erstellung der schriftlichen Ausarbeitung & X & X\\
\hline
\end{tabular}
\end{center}


    \section*{Zusammenfassung}
Die deutsche Zusammenfassung. Halbe Seite.
 

\mainmatter % die eigentliche Arbeit

%% TODO der von der Fh vorgegebene aufbau ist 
%% aufgabensstellung Thema
%% gliederung der aufgabe mit angaber der Seitenzahl
%% Textteil
%% zusammenfassung der einzelergebnisse
%% zusammenfassung aller formelzeichen 
%% Literaturverzeichniss

    
% Diesen Part des Dokuments als letztes schreiben


\chapter{ Einleitung }
\vspace{-3cm}
\begin{flushleft}
% 
% Die Einleitung schreibt man zuletzt, wenn die Arbeit im Großen und Ganzen schon fertig ist.
% (Wenn man mit der Einleitung beginnt - ein häufiger Fehler - braucht man viel länger und wirft sie später doch wieder weg). 
% Sie hat als wesentliche Aufgabe, den Kontext für die unterschiedlichen Klassen von Lesern herzustellen. Man muß hier die Leser für sich gewinnen. 
% Das Problem, mit dem sich die Arbeit befasst, sollte am Ende wenigsten in Grundzügen klar sein und dem Leser interessant erscheinen. 
% Das Kapitel schließt mit einer Übersicht über den Rest der Arbeit. Meist braucht man mindestens 4 Seiten dafür, mehr als 10 Seiten liest keiner.
% 
% Diese fragen Sollten beantwortet werden:
% Welche Fragestellung will ich behandeln ?
% Warum ist sie wichtig ?
% Warum ist sie nicht trivial ? ( bereits gelöst oder einfach lösbar )
% was möchte ich neues zur Lösung der Fragestellung beitragen ?
% Welche empirische Behandlung wähle ich zur Lösung der Fragestellung ?

\section{Motivation}
\label{sec:Motivation}
Im Rahmen einer Projektarbeit an der Fachhochschule Stralsund soll eine Methode entwickelt werden, um die innerhalb der Fachhochschulgebäude 
an den Türen angebrachten Schildern durch eine elektrische Lösung zu ersetzen. Dabei besteht besteht die Hauptaufgabe darin den Schildern einen größeren Funktionsumfang zu geben und die dafür notwendige Anzahl an Arbeitsschritten so gering wie mögliche zu halten.  

\section{Bedeutung der Arbeit}
\label{sec:Bedeutung der Arbeit}
Die bisher an den Türen angebrachten Schildern werden hauptsächlich dafür genutzt um die Belegungspläne des jeweiligen Raumes oder um wichtige Informationen von Professoren zu veröffentlichen. Dabei ist darauf zu achten, dass nur ein begrenzt nutzbarer Raum für Informationen bereitsteht. Dadurch kommt es oft dazu, dass keine neuen Bekanntmachungen an den Schildern angebracht werden können und diese so verloren gehen, da sie nur in einem eingeschränkten Zeitraum aktuell sind. Dieser Mangel im Informationsfluss kann somit auch dazu führen, dass den Studenten Möglichkeiten entgehen, die ihre Studienzeit oder auch die erste Zeit im Berufsleben beeinflussen könnten. Beispiele dafür währen die Mitteilungen von Studien- oder Abschlussarbeiten an der Fachhochschule selbst oder in der freien Wirtschaft. 
Als nächster Punkt kann die Betreuung mehrerer Räume aufgeführt werden. Es muss in der derzeitigen Situation jeder Raum einzeln betreut werden, was eine vermehrte Zahl von Arbeitsschritten verlangt. Wird allerdings eine zentrale Lösung zur Verwaltung geschaffen, kann so die Zahl der nötigen Verwaltungsvorgänge verringert werden. 

\section{Zielsetzung}
\label{sec:Zielsetzung}
Mit dieser Arbeit soll eine Möglichkeit geschaffen werden, um die oben genannten Missstände zu beheben. Diese besteht darin die derzeitigen Türschilder durch eine elektrische Alternative zu ersetzen. Diese muss daraufhin so konfiguriert werden, dass sie eine Möglichkeit bietet zentral verwaltet zu werden und muss jedoch gleichzeitig gegen unbefugte Eingaben von Dritten geschützt werden. 

\end{flushleft}

    \chapter{ Grundlagen }\label{chap:Grundlagen}
\vspace*{-3cm}
\begin{flushleft}

In diesem Abschnitt der Arbeit, werden verschiedene Begriffe genauer erläutert, die benötigt werden, um die später beschrieben Bearbeitung
des Themas nachvollziehen zu können. Dazu gehören zum einen die Definitionen der eingesetzten Geräte und die dazugehörige Software.

\section{Digitaler Bilderrahmen}
Ausgangspunkt der Arbeiten am ``intelligenten Türschild'' war ein sogenannter digitaler Bilderrahmen der Firma Kodak. Da jedoch diese digitalen Bilderrahmen oft verschiedene Ausstattungen besitzen und sich somit gravierend unterscheiden können, wird zunächst eine Definition benötigt.

Bei einem digitalen Bilderrahmen handelt es sich um ein elektronisches Gerät, welches als Haupteinsatzgebiet die Wiedergabe von Bildmaterial hat. Es können dann noch je nach Modell weitere Funktionen (z.B. die Wiedergabe von Ton- und Videomaterial) und Ausstattungsmerkmale (z.B. Lautsprecher oder Kartenlesegeräte) vorhanden sein. Jedoch handelt es sich bei diesen Geräten nur um eine meist vereinfachte Form der derzeitigen Bildschirmtechnologie (z.B. LCD-Bildschirme, welche die Anzeige mit Hilfe von Flüssigkristallen erzeugen), welche dann mit einem Rahmen aus Kunststoff oder Metall ausgestattet werden.

Neben den unterschiedlichen Hardwareausstattungen verwenden die Hersteller verschiedene Softwarelösungen, um den Nutzer die Möglichkeit zu geben dem Bilderrahmen neues Bildmaterial zur Verfügung zu stellen oder um Bilder mit anderen Personen zu teilen. Die eingesetzte Software unterscheidet sich jedoch von Hersteller zu Hersteller. So setzt zum Beispiel Kodak als Software ``Kodak Pulse'' ein. Diese wird nun im nächsten Absatz genauer dargestellt, da sie für die weitere Bearbeitung des Themas nötig ist.

\section{Kodak Pulse}
Bei der Software die auf den Namen ``Kodak Pulse'' hört, handelt es sich um einen Service den die Firma Kodak ihren Kunden zur Verfügung stellt, um von Kodak produzierte digitale Bilderrahmen zu verwalten. Dazu muss jedoch der Bilderrahmen mit einem eingebauten W-Lan-Modul ausgerüstet sein. Der Nutzer muss nun wie in der Gebrauchsanweisung beschrieben den Bilderrahmen mit dem Internet verbinden. Im nächsten Schritt muss eine Anmeldung auf der Internetpräsenz von Kodak vorgenommen werden. Im Verlaufe dieser Anmeldung wird auch der Bilderrahmen mit dem Online-Konto verknüpft. Es besteht nun die Möglichkeit über die Weboberfläche Bilder zum Speicher des Rahmens hinzuzufügen oder zu löschen. Weiterhin kann der Nutzer die im Bilderrahmen vorhandenen Bilder mit Hilfe verschiedener Internetdienste wie Facebook oder Twitter mit anderen Personen teilen.

Jedoch ist die Fortführung dieses Dienstes gefährdet. Da in jüngster Vergangenheit immer mehr finanzielle Probleme bei Kodak auftraten, ist es nun schon zum Verkauf einiger Technologien und Diensten gekommen. Auch der ``Kodak Pulse''-Dienst steht zum Verkauf. Jedoch ist noch nicht sicher ob der Dienst im Falle eines Verkaufs noch weitergeführt wird. Dieser Fakt muss in diesem Abschnitt erwähnt werden, da dies den Verlauf der Bearbeitung entscheidend beeinflusst hat. 

\section{Android}
Bei dem Softwareprodukt ``Android'' handelt es sich um ein Betriebssystem, welches vorrangig auf mobilen Geräten wie Handys bzw. Smartphones und Tablets zum Einsatz kommt. Entwickelt wird diese Software von der ``Open Handset Alliance''. Dies ist ein Zusammenschluss mehrerer Unternehmen, welche das Ziel haben offene Standards für mobile Geräte zu schaffen. Bekannte Unternehmen die diesem Verbund angehören sind Google, Ebay, Vodafone und Gerätehersteller wie HTC, LG Electronics oder Samsung Electronics. 
Als Grundlage für das Android-Betriebssystem ist der Linux-Kernel (die unterste Softwareschicht eines Betriebssystems) in der Version 2.6 herangezogen worden. Seit Android-Version 4 ist auch ein Linux-Kernel der Version 3.x möglich. 
Das Endgerät, auf dem das Android-Betriebssystem arbeitet, kann mit Hilfe sogenannter ``Apps'' (Software die auf einem Android-Gerät nachträglich installiert werden kann) welche von verschiedenen Anbietern heruntergeladen werden können,um weitere Funktionen erweitert werden. Derzeit gibt es etwas mehr als 500.000 dieser Apps (Stand August 2012). Somit kann ein Endgerät je nach eingesetzten Programmen verschiedene Aufgaben erfüllen. 

\section{ADB}
Im späteren Verlauf der Projektarbeit kommt die sogenannte ``ADB'' zum Einsatz. Diese Abkürzung steht für ``Android Debug Bridge''. Hinter diesem Begriff verbirgt sich ein Tool, welches die direkte Kommunikation mit dem Gerät ermöglicht. Es wird ausschließlich über die Kommandozeile des PCs bedient und stellt verschiedene Befehle zur Verfügung. So können zum Beispiel Dateien vom PC auf das Handy kopiert werden oder auch vom Handy gelesen werden. Weiterhin besteht auch die Möglichkeit weitere Programme mit Hilfe dieses Tools auf dem Endgerät zu installieren. Damit bietet die Android Debug Bridge die Möglichkeit, am Gerät schwer zu findende Einstellungen vom PC vorzunehmen und diese Vorgänge zu vereinfachen. 

\section{Rooten von Android} 
Unter dem Begriff ``Root'' kann man den Administrator eines Gerätes oder eines Systems verstehen. Dieser ist dabei mit den größtmöglichen Rechten innerhalb des Systems ausgestattet. Standardmäßig besteht bei Android eine Einschränkung der Rechte. Somit muss ein sogenanntes ``Rooting'' durchgeführt werden, um diese Einschränkung zu umgehen. Die Prozedur für das Rooting unterscheidet sich von Gerät zu Gerät. Hat man allerdings nun vollen Zugriff auf das Gerät können nun Einstellungen durchgeführt werden, die vorher nicht möglich waren, zum Beispiel das Installieren eines veränderten Android-Betriebssystems (sogenannte ``Custom-Roms'') oder das nutzen von Apps die tief in das System eingreifen. 
\end{flushleft}

% Hier werden zwei wesentliche Aufgaben erledigt:
% 
% Der Leser muß alles beigebracht bekommen, was er zum Verständnis der späteren Kapitel braucht. Insbesondere sind in unserem Fach die Systemvoraussetzungen zu klären, die man später benutzt. Zulässig ist auch, daß man hier auf Tutorials oder Ähnliches verweist, die hier auf dem Netz zugänglich sind.
% Es muß klar werden, was anderswo zu diesem Problem gearbeitet wird. Insbesondere sollen natürlich die Lücken der anderen klar werden. Warum ist die eigene Arbeit, der eigene Ansatz wichtig, um hier den Stand der Technik weiterzubringen? Dieses Kapitel wird von vielen Lesern übergangen (nicht aber vom Gutachter ;-), auch später bei Veröffentlichungen ist "Related Work" eine wichtige Sache.
% Viele Leser stellen dann später fest, daß sie einige der Grundlagen doch brauchen und blättern zurück. Deshalb ist es gut, Rückwärtsverweise in späteren Kapiteln zu haben, und zwar so, daß man die Abschnitte, auf die verwiesen wird, auch für sich lesen kann. Diese Kapitel kann relativ lang werden, je größer der Kontext der Arbeit, desto länger. Es lohnt sich auch! Den Text kann man unter Umständen wiederverwenden, indem man ihn als "Tutorial" zu einem Gebiet auch dem Netz zugänglich macht.
% 
% Dadurch gewinnt man manchmal wertvolle Hinweise von Kollegen. Dieses Kapitel wird in der Regel zuerst geschrieben und ist das Einfachste (oder das Schwerste weil erste).


    % Wie wird aktuell mit Türschildern gearbeitet
\chapter{ Stand }\label{chap:Stand}
\vspace{-3cm}
\begin{flushleft}
Dieses Kapitel setzt sich damit auseinander, wie aktuell die herkömmlichen Türschilder eingesetzt werden und welche möglichen Lösungen existieren, um diese zu ersetzen.

\section{Nutzung der Türschilder innerhalb der Gebäude der Fachhochschule} 
Zum aktuellen Zeitpunkt werden vor den Seminarräumen und den Laboren Metallschilder und vereinzelt Schaukästen bzw. Pinnwände eingesetzt, um Informationen und Stundenpläne zu veröffentlichen. Dabei ist für jeden Raum eine separate Betreuung nötig. Dies bedeutet, dass für jede neue Information die vor dem Raum publiziert werden soll, ein neues Dokument erstellt werden und platziert werden muss. Da wie bereits beschrieben nur ein sehr eingeschränkter Raum zur Verfügung steht, muss ebenfalls entschieden werden, welche Informationen nicht mehr benötigt werden und ausgetauscht werden können. Somit ist für einen eigentlich einfachen Vorgang ein dafür großer Arbeitsaufwand nötig. Ein weiterer Nachteil der bisherigen Lösung ist, dass durch die dezentrale Verwaltung es dazu kommen kann, dass ein und dieselbe Information mehrfach ausgehängt wird. Es kann also dazu kommen, dass an einer Stelle die Information bereits nicht mehr als aktuell angesehen wird, während es an anderer Stelle immer noch für aktuell gehalten wird. Es muss somit eine Lösung geschaffen werden, die eine zentrale Verwaltung ermöglicht um damit den Arbeitsaufwand zu verringern und gleichzeitig die Aktualität von Informationen zu verbessern. 
Für dieses Problem gibt es einen Lösungsansätze, welcher im nächsten Abschnitt genauer betrachtet wird. 

\section{Welche Technik gibt es bereits?}
Um das Problem des vermehrten Arbeitsaufwands und der doppelten Informationen zu lösen, entwickelten verschiedene Firmen sogenannte digitale Türschilder. Dabei handelt es sich meistens um besonders energiesparende PCs, welche zusammen mit einem Display in einem Gehäuse verbaut werden. Diese PCs können dann entweder über eine LAN- bzw. WLAN-Schnittstelle mit Informationen gefüllt werden. Bei manchen Modellen werden zusätzlich Kartenlesegeräte (z.B. für SD-Karten) verbaut. Dann kann mit Hilfe einer meist mitgelieferten Software die Türschilder verwaltet Dokumente auf diese übertragen werden. 
Nachteil dieser Lösung ist jedoch, dass diese Türschilder ausschließlich für dieses einzigen Einsatzzweck konzipiert wurden. Es gibt also keine Möglichkeit die Geräte anderweitig einzusetzen. Daraus resultiert also ein hoher Anschaffungspreis für ein ``einfaches'' Türschild. 
Weitere Lösungen basieren auf sehr ähnlichen Prinzipien und unterscheiden sich oft nur in der Verwaltungssoftware und darin, welche Formate das Türschild wiedergeben kann. 
\end{flushleft}



    \chapter{Anforderungsdefinition} % (fold)
\label{sec:Anforderungen}
\vspace{-3cm}
\begin{flushleft}
In diesem Kapitel wird erläutert welche Ansprüche von der Universität Greifswald an die Software gestellt werden.
Dazu werden die funktionalen in Form von ``User Stories'' beschrieben und die nichtfunktionalen Anforderungen werden durch ISO/IEC 9126 (siehe \cite{ISO9126} und \cite{ISO}) beschrieben.
% Dazu werden die funktionalen Anforderungen durch ``Use cases'' beschrieben und 
% die nichtfunktionalen Anforderungen werden durch \gls{ISOSQ} beschrieben.


\section{Funktionale Anforderungen} % (fold)
\label{sec:Funktionale Anforderungen}

\subsection{User Stories}
\label{sec:Usecases}
\begin{flushleft}
% eine einleitung wenn hier viel text folgt.
In diesem Abschnitt soll die geplante Verwendung der Software erklärt werden.
Dafür werden die verschiedenen User Stories vorgestellt.

% warum kommt es zum einsatz ?
% Eine Softwarelösung für das Problem der Zahnbewegung soll hauptsächlich für zwei Dinge erfolgen.
% Zum einen soll die Software für die Lehre benutzt werden und zum anderen um Voraussagen für konkrete Behandlungen zu treffen.

% Die ``Use cases'' für diese Softwareentwicklung sind im Appendix Abschnitt \ref{sec:Use cases} auf Seite \pageref{sec:Use cases} zu finden, 
% werden hier aus Übersichtlichkeisgründen aber kurz erklärt.

\subsubsection{Daten aus HIS-LSF auslesen} % (fold)
\label{ssub:Daten aus HIS-LSF auslesen}
% Es sollen aus den Datenbeständen von \acs{HIS}-\acs{LSF} Informationen ausgelesen werden, die dann weiter verarbeitet werden und mit \acs{Moodle} abgeglichen werden. 
Als User möchte ich Informationen aus den Datenbeständen von \acs{HIS}-\acs{LSF} auslesen, um diese an die Datenverarbeitung zu übertragen.

\subsubsection{Daten aus Moodle auslesen}
\label{ssub:Daten aus Moodle auslesen}
% Es sollen aus den Datenbeständen von \acs{Moodle} Informationen ausgelesen werden, die dann weiter verarbeitet werden und mit \acs{HIS}-\acs{LSF} abgeglichen werden. 
Als User möchte ich Informationen aus den Datenbeständen von \acs{Moodle} auslesen, um zu kontrollieren, ob schon ein User in der Datenbank von \acs{Moodle} vorhanden ist.

\subsubsection{Daten in Moodle schreiben}
\label{ssub:Daten in Moodle schreiben}
% Die Software soll aus dem \acs{HIS}-\acs{LSF} übermittelte Daten an die Geschäftslogik von \acs{Moodle} senden, damit diese die Informationen in die eigene Datenbank aufnehmen kann.
Als User möchte ich Daten an die Geschäftslogik von \acs{Moodle} senden, um einen Nutzer in die Datenbank aufzunehmen. 

\subsubsection{Kontrolle der Richtigkeit und Vollständigkeit der Daten}
\label{ssub:Kontrolle der Richtigkeit der Daten}
% Die Schnittstelle muss dazu in der Lage sein, während der Übermittlung die Daten auf Richtigkeit und Vollständigkeit zu prüfen. 
Als User möchte ich, dass vor der Übermittlung von Daten, diese kontrolliert werden, damit keine falschen Informationen in der Datenbank vorhanden sind.

\end{flushleft}



% Der Benutzer der Software soll in der Lage sein präparierte Inventor-Daten zu Laden.
% %% TODO erklärung des dateiformats ?
% Der Aufbau des Dateiformats wird im Anhang Abschnitte \ref{sec:Dateiformat} erklärt.
% Hierbei soll es dem Nutzer möglich sein die Datei durch einen Dateiauswahldialog auszuwählen um sie dann in das Programm zu laden.
% Sind die Daten geladen zeigt sich dem Nutzer eine Visuelle repräsentation der Zähne.





%TODO was einfügen
% section Funktionale Anforderungen (end)
 \newpage
\section{Nichtfunktionale Anforderungen} % (fold)
\label{sec:Nichtfunktionale Anforderungen}

\subsection{Funktionalität} % (fold)
\label{sub:Funktionalitaet}

Die Software muss dazu in der Lage sein des Qualitätsmerkmal der ``Richtigkeit'' in dem Sinne zu erfüllen, dass die Anwendung Informationen so in 
die Datenbestände einträgt, wie es ein Nutzer der Systeme tut.

Weiterhin muss die Software das Merkmal der ``Interoperabilität'' soweit erfüllen, dass sie keine Auswirkungen auf die angeschlossenen Systeme in Bezug auf 
Stabilität und Verfügbarkeit hat. 
% Die Software muss dazu in der Lage sein des Qualitätsmerkmal der ``Richtigkeit'' in dem Sinne zu erfüllen,
% dass die Anwendung vergleichbare Ergebnisse liefert wie es behandlungen an Menschen erbracht haben. (Toleranz von 1 mm Abweichung ) 
% Weitreichendere Anforderungen an Interoperabilität, Sicherheit und Ordnungsmäßigkeit müssen nicht erfüllt werden.
% subsection Funktionalität (end)

\subsection{Zuverlässigkeit} % (fold)
\label{sub:Zuverlaessigkeit}

Die Software soll über eine gewisse ``Reife'' verfügen. 
Im Speziellen heißt es, dass es nur in einen von zehn Szenarien zu Ausfällen kommt.
Die Software sollte bei einem Fehler eine Mitteilung bereitstellen, damit ein Verantwortlicher der Universität darauf reagieren kann. 
% subsection Zuverlässigkeit (end)

\subsection{Benutzbarkeit} % (fold)
\label{sub:Benutzbarkeit}

Die Software muss einfach verständlich sein ( ``Verständlichkeit'' ) und sich einfach Erlernen lassen (``Erlernbarkeit'').
Eine einfache Benutzung steht dabei im Vordergrund ( ``Benutzbarkeit'' ), hingegen soll ``Attraktivität'' nur soweit gewährleistet sein,
dass sie die ``Verständlichkeit'' positiv beeinflusst. 

Es ist jedoch auch möglich, dass die Software ohne den Einfluss eines Nutzers arbeitet. Dann sind die oben genannten Punkte nicht relevant. 

% subsection Benutzbarkeit (end)

\subsection{Effizienz} % (fold)
\label{sub:Effizienz}

Das ``Verbrauchsverhalten'' der Softwarelösung sollte so sein, dass diese die Server so wenig belastet wie möglich.
Da jedoch noch kein Server definiert wurde, auf dem die Anwendung arbeiten soll, können keine Werte für eine genaue Konfiguration
angegeben werden.
% Das ``Verbrauchsverhalten'' der Softwarelösung sollte so sein, 
% dass das fertige Programm auf einem durchschnittlichen Computer ( CPU: 2,6 GHz RAM: 2Gb ) lauffähig ist.
% Dabei ist die Antwortzeit des Systems sollte bei einer durchschnittlichen Maschine nicht länger als 0.5 sek betragen.
% subsection Effizienz (end)

\subsection{Wartbarkeit} % (fold)
\label{sub:Wartbarkeit}

Es soll die Möglichkeit bestehen, leicht neue Funktionen zur Software hinzuzufügen ohne den konzeptionellen Aufbau der Software zu verändern. 
% subsection Wartbarkeit (end)

\subsection{Übertragbarkeit} % (fold)
\label{sub:Übertragbarkeit}

Auf die Qualitätsanforderung ``Übertragbarkeit'' soll in dem Sinne eingegangen werden, dass die Software auf Linux und Windows lauffähig ist ( ``Anpassbarkeit'' ).

% subsection Übertragbarkeit (end)
% section Nichtfunktionale Anforderungen (end)

% chapter Anforderungsdefinition (end)
 
\end{flushleft}


    \chapter{ Entwurf }
\vspace{-3cm}
\begin{flushleft}
Dieses Kapitel beschäftigt sich damit, welche Ziele die Arbeit verfolgt und wie diese mit den Anforderungen an das Endprodukt zusammengebracht werden. Weiterhin werden verschiedene Konzepte aufgezeigt, welche als Möglichkeit zur Problemlösung zur Verfügung standen. Diese werden daraufhin mit den Zielen und Anforderungen verbunden und es wird begründet, welche Lösung gewählt wurde.

\section{Ziele der Arbeit}\label{sec:Ziele der Arbeit}
Wie bereits beschrieben, verfolgt dieses Projekt verschiedene Intentionen. Diese werden an dieser Stelle noch einmal kurz aufgegriffen, um ein besseres Verständnis der folgenden Abschnitte zu gewährleisten.
\newline

Die Zielstellungen des Projekts lauten wie folgt:
\begin{itemize}
  \item Schaffung einer Lösung die bisherigen Türschilder zu ersetzen
  \item Verminderung des Verwaltungsaufwandes zur Eingabe und Pflege der zu publizierenden Informationen
\end{itemize}

Es muss allerdings berücksichtigt werden, dass diese Ziele nicht ohne Einschränkungen verfolgt werden können, denn sie sind an Anforderungen gebunden. Diese wurden bereits im Abschnitt \ref{sec:Anforderungen} ``Anforderungsdefinition'' vorgestellt.

\section{Konzepte zur Problemlösung}
Die in den Abschnitten \ref{sec:Ziele der Arbeit} und \ref{sec:Anforderungen} beschrieben Punkte müssen nun miteinander in Verbindung gebracht werden und es müssen, unter Berücksichtigung dieser Aspekte, mögliche Konzepte ausgearbeitet werden.
Diese Ausarbeitung und deren Ergebnisse werden in den folgenden Passagen genauer vorgestellt. Dabei werden zuerst verschiedene Lösungsansätze präsentiert, die dann bewertet werden. Im Anschluss daran wird einer der Vorschläge als Lösungsansatz gewählt und im nächsten Kapitel genauer betrachtet.

\subsection{Grundidee}
Für die Umsetzung des Projekts ist das bereits vorhandene Grundkonzept des energiesparenden PCs welcher mit einem Bildschirm in ein Gehäuse verbaut ist, die beste Grundlage. Jedoch ist es nicht möglich, die bereits fertigen Lösungen einzusetzen, da diese sehr kostenintensiv und unflexibel sind. Somit muss eine Abwandlung dieses Konzeptes entwickelt werden, um es für die Fachhochschule Stralsund einsatzfähig zu machen. 

\subsection{Nutzung eines digitalen Bilderrahmens als Türschild}
Die erste hier gezeigte Umsetzung ist es, die Grundidee mit Hilfe eines digitalen Bilderrahmens umzusetzen. 


\end{flushleft}

% st das zentrale Kapitel der Arbeit. Hier werden das Ziel sowie die eigenen Ideen, Wertungen, Entwurfsentscheidungen vorgebracht. Es kann sich lohnen, verschiedene Möglichkeiten durchzuspielen und dann explizit zu begründen, warum man sich für eine bestimmte entschieden hat. Dieses Kapitel sollte - zumindest in Stichworten - schon bei den ersten Festlegungen eines Entwurfs skizziert und in Stichworten geschrieben werden. Es wird sich aber in einer normal verlaufenden Arbeit dauernd etwas daran ändern. Das Kapitel darf nicht zu detailliert werden, sonst langweilt sich der Leser. Es ist sehr wichtig, das richtige Abstraktionsniveau zu finden. Bei der Verfassung sollte man auf die Wiederverwendbarkeit des Textes achten.
% 
% Plant man eine Veröffentlichung aus der Arbeit zu machen, können von diesem Kapitel Teile genommen werden. Das Kapitel wird in der Regel wohl mindestens 8 Seiten haben, mehr als 20 können ein Hinweis darauf sein, daß das Abstraktionsniveau verfehlt wurde.


    \chapter{ Umsetzung }\label{chap:Umsetzung}
\vspace*{-3cm}
\begin{flushleft}
%Hier greift man einige wenige, interessante Gesichtspunkte der Implementierung heraus. Das Kapitel darf nicht mit Dokumentation oder gar Programmkommentaren verwechselt werden. Es kann vorkommen, dass sehr viele Gesichtspunkte aufgegriffen werden müssen, ist aber nicht sehr häufig. Zweck dieses Kapitels ist einerseits, glaubhaft zu machen, daß man es bei der Arbeit nicht mit einem "Papiertiger" sondern einem real existierenden System zu tun hat. Es ist sicherlich auch ein sehr wichtiger Text für jemanden, der die Arbeit später fortsetzt. Der dritte Gesichtspunkt dabei ist, einem Leser einen etwas tieferen Eindruck in die Technik zu geben, mit der man sich hier beschäftigt. Schöne Bespiele sind "War Stories", also Dinge mit denen man besonders zu kämpfen hatte, oder eine konkrete, beispielhafte Verfeinerung einer der in Kapitel 3 vorgestellten Ideen. Auch hier gilt, mehr als 20 Seiten liest keiner, aber das ist hierbei nicht so schlimm, weil man die Lektüre ja einfach abbrechen kann, ohne den Faden zu verlieren. Vollständige Quellprogramme haben in einer Arbeit nichts zu suchen, auch nicht im Anhang, sondern gehören auf Rechner, auf denen man sie sich ansehen kann.
% hier könnte drin stehen:
% Versuch mit dem PHP-Skript für Kodak-Pulse und Router (DNS-MASQ, SSL-Strip)
% Entwicklung der Anwendersoftware und Androidapp (welche Funktionen brauch die app, wie eingepflegt, energieprobleme?)
\section{Versuche mit dem Kodak-Pulse Bilderrahmen}
  Trotz dessen das Projekt letztendlich nicht mit den Kodak-Pulse Bilderrahmen umgesetzt wurde, sollen hier die wichtigsten Schritte bei dem Versuch ihn zu verwenden erläutert werden, da es ein wesentlicher Bestandteil des Projektes war.
  \subsection{Anfragen des Kodak-Pulse Bilderrahmen umleiten}
    Als erste Idee, sollte ein lokaler Webserver auf die Domain \textbf{device.pulse.kodak.com} reagieren, wo drauf, das im Abschnitt \ref{subsec:nutzungBilderrahmen} \nameref{subsec:nutzungBilderrahmen} erwähnte PHP-Skript liegt. Dies sollte die korrekten Antworten an den Bilderrahmen zurückliefern.
    Dazu benötigten wir einen Router inkl. \textit{Dnsmasq}\footnote{Dnsmasq ist ein sehr einfacher, kombinierter DNS- und DHCP-Server für kleine bis mittlere Netzwerke, der eine leicht verständliche Konfiguration ermöglicht, und sehr zuverlässig arbeitet. Mithilfe dieses Tools ist es möglich einen gewählten FQDN auf eine vom Anwender selbst gewählte IP-Adresse zuleiten.}, einen Rechner der als Webserver fungiert und natürlich den Bilderrahmen.
    \begin{enumerate}
      \item Zuallererst müssen der Bilderrahmen und der Rechner so konfiguriert werden, das sie eine Netzwerkverbindung mit dem Router herstellen können. Dabei ist zu beachten, dass der Rahmen und der Rechner als DNS-Server und Gateway die IP-Adresse des Routers eingetragen haben.
      \item Auf dem Rechner, mit dem Betriebssystem Ubuntu 11.10, wurde ein Apache-Webserver installiert. Die angezeigten Webseiten des Webservers werden über sogenannte Virtualhost-Konfiguration eingerichtet. Solch eine Konfiguration wurde so angepasst, dass das PHP-Skript, für alle Anfragen an den Webserver, aufgerufen wird. Dies wurde in der standardmäßig auf dem Webserver vorhandenen Virtualhost-Konfiguration mithilfe von einer \textit{RewriteRule} gemacht, diese leitet nun alle Anfragen entsprechend des regulären Ausdrucks an das PHP-Skript weiter. Außerdem muss der Apache nicht nur auf Port 80 und sondern auch auf Port 443 hören, weil der Rahmen einiges über das SSL-Protokoll verschlüsselt senden will. Der Port 80 ist der Standard HTTP-Port, auf diesen hört der Apache schon standardmäßig. Außerdem muss er nun auf Port 443 hören der Standard HTTPS-Port, dazu einfach die vorhandene Virtualhost-Konfiguration kopieren und an das Ende der bestehenden Datei anhängen. Nun muss nur noch der Port 80 mit Port 443 ersetzt werden.\\\vspace{.3cm}
      \textbf{RewriteRule:}
      \verb|RewriteRule /DeviceRest.* /kodak-pulse-picture-frame-server.php|\\
      \vspace{.5cm}
      \textbf{Port 80/443:}\\
      \verb|<VirtualHost *:80> ... </VirtualHost>|
      \verb|<VirtualHost *:443> ... </VirtualHost>|
      \item Nun kommunizieren der Rahmen und der Rechner schon mit dem Router. Außerdem ist der Webserver soweit korrekt konfiguriert. Was noch fehlt ist das der Webserver auf die FQDN \textbf{device.pulse.kodak.com} reagiert, da der Bilderrahmen alle seine Anfragen an diese Domain übermitteln will. Daher muss nun auf dem verwendeten Router das Tool \textit{Dnsmasq} benutzt werden. Hat man dieses eingeschaltet sucht der Dienst in der Systemdatei \textit{/etc/hosts} nach Einträgen, dort kann man angeben was für eine FQDN zu welcher IP-Adresse gehört. Es ist sozusagen ein statischer/manueller DNS-Server.\\\vspace{.3cm}
      \textbf{Beispieleintrag in /etc/hosts:}\\
      \verb|192.168.0.1 	device.pulse.kodak.com|
    \end{enumerate}
    Damit sollte der Bilderrahmen sich nun eigentlich am angelegten Webserver Authentifizieren können und man somit Bilder auf den Bilderrahmen bringen kann. Da die Firmware auf dem verwendeten Rahmen, aber nun doch den Hostnamen mit dem SSL-Zertifikat vergleichen will, klappt dies leider nicht. Da erstens der Hostnamen nicht mit dem SSL-Zertifikat übereinstimmen kann und zweitens hat der Server gar kein Zertifikat hinterlegt. Es wurde auch versucht verschiedene SSL-Zertifikate zu hinterlegen, was aber nicht den gewünschten Erfolg brachte. Das führte zu einer weiteren Überlegung ein Tool namens SSL-Strip zu nutzen.
  \subsection{Umgeleitete Anfragen durch das Tool SSL-Strip schicken}
    Um mit dem Tool SSL-Strip zu arbeiten ist die vorher beschriebene Konfiguration ebenfalls notwendig. Hinzu kommt nun in unserem Fall noch eine Virtuelle Maschine. Diese wurde mit einem vorgefertigten Image namens BackTrack erstellt. Dieses Linux-Image enthält viele Tools, um ein Netzwerk zu ``Sniffen''. Unter anderem enthält die VM auch das Tool SSL-Strip.
    
    Das Tool dient dazu einen empfangenen Netzwerkverkehr auf HTTPS Pakete zu durchsuchen und die mit HTTP zu ersetzen, somit ist es möglich z.B. GMAIL oder auch PAYPAl Anmeldedaten im Klartext mit zu lesen. Außerdem, was für unseren Einsatz wesentlich wichtiger ist, werden die Pakete so manipuliert weitergeschickt. Dadurch haben wir vermutet können wir den Bilderrahmen austricksen, was aber letztendlich auch nicht zum Erfolg führte.
    
    Die VM muss nun natürlich erstmal so konfiguriert werden das sie im gleichen Adressbereich wie der Router und Bilderrahmen hängt. Daraufhin muss nun die VM darauf vorbereitet werden den IPv4 Netzwerkverkehr komplett zu empfangen und auch wieder weiterzuleiten (IPv4-Forwarding).\par Wie man vllt. erkennen kann, versuchen wir jetzt einen Man-in-the-Middle Angriff durchzuführen, um somit die SSL-Verschlüsselung des Rahmens zu umgehen.\vspace{.3cm}
    
    \underline{Vorgehen:}\\
    \begin{itemize}
    \item Als erstes müssen wir IPv4-Forwarding aktivieren. Das geht unter Linux sehr einfach, in der Datei ``/proc/sys/net/ipv4/ip\_forward'' steht im normalfall eine ``0'' d.h. das IPv4-Forwarding deaktiviert ist. An diese Stelle muss nun eine ``1'' um das Forwarding zu aktivieren. Dazu schreibt man entweder mit einem Rootfähigen Editor eine ``1'' in die Datei bzw. führt mit Rootrechten in einem Terminal folgenden Befehl aus, der ebenfalls die ``1'' in die Datei schreibt.\\\vspace{.3cm}
    \begin{center}
      \verb+echo 1 > /proc/sys/net/ipv4/ip_forward+
    \end{center}    
    \item Daraufhin muss man sicherstellen, dass die Pakete vom Bilderrahmen an den Webserver auch den Umweg über die VM gehen. Dazu nutzen wir das Tool ``ARPSPOOF'', dieses ermöglicht es manipulierte ARP-Nachrichten zu verschicken und somit eine IP-Adresse eine andere MAC-Adresse zuzuweisen. Es muss nun eine Nachricht gesendet werden die dem Bilderrahmen bzw. dem Webserver vortäuscht das die Gateway IP des Netzes durch das die Pakete um jeden Preis durch wollen, die MAC-Adresse der VM mit SSL-Strip hat. Dadurch geht der Verkehr nun durch die VM. Folgend ein beispielhafter Aufruf:
    \begin{center}
      \verb+arpspoof -i eth0 -t 192.168.1.6 192.168.1.1+
    \end{center}
    \item Wenn man SSL-Strip startet hört es standardmäßig auf Port 10000, der dort antreffende HTTP/HTTPS Verkehr wird daraufhin manipuliert. Dazu benötigen wir also ein Route die den Verkehr auf Port 80 auf den Port 10000 umleitet. Dazu wird ein weiteres Tool IPTables genutzt, hiermit lassen sich sehr schnell komplexe Netzwerkrouten erstellen. Um nun das hier gewünschte Verhalten zu erreichen müssen wir folgenden Befehl ausführen.
    \begin{center}
      \verb+iptables -t nat -A PREROUTING -p tcp \+
      \verb+--destination-port 80 -j REDIRECT --to-ports 10000+
    \end{center}
    \item Nun ist die VM fertig vorbereitet, somit kann man zum Abschluss nun das SSL-Strip Tool starten. Man kann nun im laufenden Betrieb in den Logdaten sehen das der Verkehr manipuliert wird. Das Tool starten gelingt mit folgenden Befehl:
    \begin{center}
      \verb+python sslstrip.py+
    \end{center}
    \end{itemize}
    Somit wird nun der Netzwerkverkehr von HTTPS Anfragen gesäubert, was zu dem Erfolg führen sollte, dass der Rahmen nun das Zertifikat nicht prüfen will. Die Firmware des Rahmens lies sich, aber auch auf diese Weise nicht austricksen.
    
    Dadurch haben wir aus schon erwähnten Gründen gegen die Bilderrahmen entschieden.

\section{Beschreibung der Software zur Notizenerstellung und der Tabletverwaltung am PC}
Um den Nutzer später die Möglichkeit zu geben kleine Notizen zu erstellen und diese dann auf das elektronische Türschild zu übertragen, wurde eine Software auf Java-Basis geschaffen, die diese Funktionalität bietet. Weiterhin kann das Tablet damit auch auf den Einsatz als Türschild vorbereitet werden, das heißt es kann gerootet werden, die nötige Software kann darüber installiert werden und das aktivieren bzw. das deaktivieren der Hardwarebuttons ist damit möglich. 
Um nun also das Tablet für seinen späteren Einsatzzweck einzurichten, stellt die Software vier Buttons bereit. Diese werden nun im einzelnen genauer vorgestellt. 
Da ein nötiger Schritt das rooten des Gerätes ist, bietet die Software an, dies per Knopfdruck zu vollziehen. Wird also nun dieser Button betätigt, so wird im Hintergrund zunächst erkannt, auf welchem Betriebssystem das Programm ausgeführt wird. Dementsprechend wird dann entweder ein Batch-Skript (Windows) oder ein Bash-Skript (Linux) ausgeführt. Dabei wird eine ADB-Verbindung zum Tablet hergestellt und es werden auf dem Gerät die nötigen Einstellungen vorgenommen, z.B. das ersetzen von Systemdateien oder das ändern von Besitzrechten. Nach einem automatischen Neustart ist dann das Odys Xelio gerootet und es kann mit dem nächsten Schritt fortgefahren werden.

Dabei handelt es sich um das installieren aller nötigen Applikationen (kurz Apps) die für die Verwendung als Türschild nötig sind. Mit einem Klick auf den Button ``Software installieren'' beginnt der Installationsvorgang automatisch. Es wird wie schon beim rooten zuerst das ausführende Betriebssystem geprüft und danach entschieden, welches Batch- bzw. Bash-Skript gerufen wird. Es kommt dann zunächst zu einer Prüfung, ob alle nötigen Installationsprogramme (.apk-Dateien) vorhanden sind. Sollte dies nicht so sein, bricht der Vorgang ab. Ist die Prüfung jedoch erfolgreich abgeschlossen, wird eine ADB-Verbindung zum Gerät hergestellt und mit dem Befehl ``adb install *.apk'' die Software installiert (der Stern wird dabei mit dem Namen des Programms ersetzt). Das zuvor erschienene Konsolenfenster kann nun mit einem Tastendruck beendet werden und der Vorgang ist abgeschlossen. 

Im letzten Schritt müssen nun noch die Hardwarebuttons des Odys Xelio deaktiviert werden, damit keine Navigation mehr möglich ist. Auch dies ist mit der Software möglich. Der Vorgang wird mit dem Button ``Hardwarebuttons deaktivieren'' gestartet. Es wird erneut zuerst geprüft, ob alle nötigen Dateien vorliegen. Sollte dies so sein, wird wieder die ADB-Verbindung aufgebaut. Um nun die Hardwaretasten zu deaktivieren, wird die Datei ``sun4i-keyboard.kl'', welche sich in ``/system/usr/keylayout'' befindet, gegen eine veränderte Version ausgetauscht. Bei dieser Datei handelt es sich um eine Definition der Ereignisse die ausgelöst werden, sobald eine Taste gedrückt wird, zum Beispiel das Wechseln auf den Homescreen, das Zurückgehen oder das Öffnen des Menüs. Jedoch kann mit mit Hilfe des Ereignisses ``FOCUS'' eine Taste auch deaktiviert werden. Im Falle des Odys Xelio geschieht genau das. Alle Tasten bekommen dieses Ereignis zugewiesen und werden so deaktiviert. Um die Veränderungen abzuschließen, muss dann das Gerät neu gestartet werden. Um die Tasten dann wieder aktivieren zu können, muss in der Software der Button ``Hardwarebuttons aktivieren'' gedrückt werden. Es geschieht intern das selbe wie beim deaktivieren, jedoch wird wieder die originale Version der Datei ``sun4i-keyboard.kl'' eingefügt. 
Damit wäre das Tablet mit Hilfe des PCs konfiguriert. Alle weiteren Einstellungen müssen nun am Tablet selbst vorgenommen werden.

Die letzte Funktionalität der Software ist das erstellen von kleinen Notizen. Dafür kann der Nutzer eine von ihm im Vorhinein erstellte Vorlage für den Hintergrund wählen. Zu beachten ist dabei, dass die Mitte frei bleibt, denn dort wird der Text eingefügt. Ist diese ausgewählt, muss jetzt noch die Nachricht eingegeben und die Schriftgröße ausgewählt werden. Die Nachricht wird nun mit Hilfe der Methode ``drawString'' der Java-Klasse ``Graphics2d'' gerendert. Die Methode benötigt neben dem zu schreibenden String noch die Koordinaten in x- und y-Richtung. In der vorliegenden Software wird dabei immer der Bildmittelpunkt bestimmt. Dies geschieht mit Hilfe der Methoden ``getHeight()'' und ``getWidth()'' der ``Bufferedimage''-Klasse. Jedoch muss nur die y-Koordinate berechnet werden, da der Bildmittelpunkt in x-Richtung immer 0 entspricht. Für die y-Koordinate wird zunächst der Bildmittelpunkt in y-Richtung mit ``buffImg.getHeight/2 bestimmt.
Davon wird allerdings noch die Höhe der Schrift abgezogen. Diese kann mit Hilfe von ''g2d.getFontMetrics().getHeight())`` bestimmt werden. Dabei handelt es sich bei ''g2d`` um ein Graphics2d-Objekt. Diesem wird mit der Methode ''setFont()`` ein ''Font``-Objekt übergeben, welches die Schriftart, den Schriftstil und die Schriftgröße enthält. Anhand dieser Parameter kann nun auch die y-Koordinate für den Mittelpunkt ermittelt werden und das Bild kann erstellt werden.

\section{Beschreibung der IntelligentDoorplateControl Applikation zur Einrichtung des Tablets}
Diese Applikation soll zum ersten die Einrichtung als Türschild direkt auf dem Tablet erleichtern. Dazu lassen sich mittels der Applikation die anderen Apps starten die zu Nutzung als Türschild erforderlich sind. Des weiteren sperrt die Applikation nach gewisser Zeit den Touchscreen, um die Nutzung Unbefugter vorzubeugen. Außerdem wird mit ihrer Hilfe die Statusbar versteckt um das volle Display nutzen zu können. Und die Applikation trägt auch ihren Beitrag zur Green-IT bei, in dem sie das Tablet in einem bestimmten Zeitraum in einen Standbymodus versetzt.

Die Applikation ist im Quellcode dabei also eher recht simpel gehalten, da sie überwiegend nur als eine Art Oberfläche dient. Dabei gibt es eine Activity die alle Schaltflächen der App beinhaltet, außerdem gibt es noch eine Klasse namens ``Device'' diese wurde dem Projekt \textbf{HideBar}\footnote{Dieses Projekt ist selbst eine Android-Applikation die es ermöglicht die Statusbar zu manipulieren. Der Quellcode liegt offen auf Github, da für uns nur das verstecken der Leiste entscheidend war, haben wir anstatt die Applikation selbst zu nutzen, nur eine Klasse des Projekt in Verwendung.\url{http://ppareit.github.com/HideBar/}} entnommen und dient zum verstecken der Statusbar des Tablets. Um den Standbymodus und das blockieren des Touchscreens Zeitgesteuert zu verwalten, gibt es noch einen Service der eine BroadcastReceiver jede Minute aufruft. Wobei im Service ein partieller Wakelock aktiviert wird, damit das Tablet nie in den Tiefschlaf fällt. Im Receiver wird nach den ersten 5 Minuten der Touchscreen blockiert und ausserdem je nach Angaben des Zeitraums für den Standbymodus, das Tablet mit Hilfe der App ``ScreenOffAndLock'' in den Standbybetrieb versetzt und mithilfe eines FullWakelocks in normal betrieb versetzt.

%Ich würde jetzt hiernach noch deine App erklären und dann die Überleitung zum nächsten Abschnitt machen.
\end{flushleft}
    \chapter{ Ausblick }\label{chap:Ausblick}
\vspace*{-3cm}
\begin{flushleft}
In diesem Kapitel wird ein kurzer Überblick gegeben, welche Erweiterungen noch für das Projekt möglich wären und erste Ansätze wie diese umgesetzt werden könnten. 

\section{Wireless-ADB}
Im bisherigen Verlauf des Projektes werden viele Einstellungen (wie z.B. das Rooten oder die Software-Installation) über ADB vorgenommen. Nachteil an dieser Methode ist, dass das Tablet jedes mal per USB-Kabel an einen PC angeschlossen werden muss. Dies bedeutet, dementsprechend, dass diese von der Wand abmontiert werden müssen, um neue Einstellungen vorzunehmen. Eine Lösung dafür wäre, dass die ADB-Befehle nicht mit Hilfe eines USB-Kabels übertragen werden, sondern mit Hilfe einer WLan-Verbindung. Dies wird jedoch standardmäßig nicht vom Odys Xelio unterstützt. Abhilfe verschaffen jedoch Apps, welche diese Funktion nachrüsten. Diese sind im ``Google Play Store'' (bezeichnet eine Plattform, welche das herunterladen und installieren von Apps auf Android Geräten ermöglicht) unter dem Begriff ``Wireless ADB'' zu finden. Somit ist der Nutzer nicht mehr dazu gezwungen ein USB-Kabel für die Konfiguration zu nutzen, sondern kann alle Einstellungen über Wlan vornehmen, solange sich das Tablet und der Router in einem Netzwerk aufhalten und die entsprechende App aktiviert ist. Somit ist nun auch nicht mehr nötig das Tablet für Einstellungen von der Wand abzunehmen. 
Jedoch ist es nun nicht einfach möglich die zum Projekt mitgegebenen .batch- bzw. .bash-Dateien zu nutzen, denn diese geben keinen Aufschluss darüber, wo sich das Gerät im Netzwerk befindet. Eine Möglichkeit dieses Missstand zu beheben, wäre es für jedes zu verwaltende Tablet einen angepassten Satz .batch- bzw. .bash-Dateien anzulegen. Da dies jedoch den Verwaltungsaufwand stark erhöhen würde, muss eine andere Lösung her. Ein Ansatz wäre somit die dynamische Erzeugung der Dateien. Dafür muss das beigelegte Verwaltungsprogramm, welches unter anderem für das Rooten oder das aktivieren bzw. deaktivieren der Hardwarebuttons verantwortlich ist, erweitert werden. Bei dieser Erweiterung handelt ist sich zum einen um eine Möglichkeit eine gültige IP-Adresse einzugeben (dieses Feld sollte mit einem Radio-Button aktiviert werden, um eine bessere Unterscheidung zwischen den beiden Methoden zu ermöglichen) und zum anderen die dynamische Erstellung von entsprechenden Dateien. Da bei der WLan-Verbindung zu einem Tablet eine jedes mal eine andere IP-Adresse vorhanden sein kann, muss für jede Aktion eine neue Datei geschrieben und daraufhin ausgeführt werden. Wie das Schreiben einer Datei aussehen könnte, zeigt folgendes Listing (gezeigt am Beispiel des Rootens):
\newpage
\begin{lstlisting}[caption={Dynamische Erstellung einer .batch-Datei}\label{dynamische Dateierstellung},captionpos=t,linewidth=\textwidth,language=JAVA] 
 
//erstellen der zu schreibenden .bat-Datei
File file = new File(System.getProperty("user.dir") + "\\ADB\\wireless_mkroot.bat");
try{
//Datei wird immer neu ueberschrieben
FileWriter writer = new FileWriter(file);	

//es wird abgefragt, ob ein Radio-Button mit Namen "rdbtnOnline" aktiviert ist
//falls ja: die geforderte Einstellung wird mit Wireless-ADB ausgefuehrt
//falls nein: die geforderte Einstellung wird mit kabelgebundenen ADB ausgefuehrt
if(rdbtnOnline.isSelected() == true)
{
	//es wird abgefragt, ob das Feld fuer die IP ausgefuellt wurde
	if(IPFeld.getIP() != null)
	{
		//es wird der noetige string in einen zuvor erstellten FileWriter geschrieben
		//dabei unterscheidet sich diese Anweisungen in zwei Punkten
		//von der USB-Variante:
		//erster Punkt:
		//"adb disconnect " + IPFeld.getIP() + "\n" +
		//dieser Befehl fuehrt zur Sicherheit eine Trennung des Geraetes 
		//mit der ausgelesenen IP durch
		//zweiter Punkt:
		//"adb connect "  + IPFeld.getIP() + "\n" +
		//dieser Befehl sorgt nun dafuer, dass eine Wlan-Verbindung zum Tablet
		//hergestellt wird und ADB-Befehle gesendet werden koennen 
		//alle weiteren Kommandos entsprechen der USB-Variante
		writer.write(
				"@echo off\n" +
				"setlocal\n" +
				"set version=0.3\n" +
				"title %~nx0 - version %version% - by GK\n" +
				"mode CON: COLS=100 LINES=44\n" +
				"color 1E\n\n" +
				
				"cd ADB\n\n" +
				
				"if not exist su (\n" +
				"echo ERROR: file 'su' not found in %~dp0 !\n" +
				"goto END\n" +
				")\n\n" +

				"if not exist Superuser.apk (\n" +
				"echo ERROR: file 'Superuser.apk' not %~dp0 !\n" +
				"goto END\n" +
				")\n\n" +
				  
				"adb disconnect " + IPFeld.getIP() + "\n" +
				
				"adb connect "  + IPFeld.getIP() + "\n" +
				
				"echo Waiting for device ...\n" +
				"adb wait-for-device\n" +
				"echo Restarting ADB with root permission ...\n" +
				"adb root\n" +
				"echo Remounting /system RW ...\n" +
				"adb remount\n" +
				"echo Copying su to /system/bin/ ...\n" +
				"adb push su /system/bin/\n" +
				"echo Copying Superuser.apk to /system/app/ ...\n" +
				"adb push Superuser.apk /system/app/\n" +
				"echo Setting permission for /system/bin/su\n" +
				"adb shell /system/bin/busybox chmod 6755 /system/bin/su\n" +
				"REM echo Setting permission for /system/app/Superuser.apk\n" +
				"REM adb shell /system/bin/busybox chmod 6755 /system/app/Superuser.apk\n" +
				"echo Removing existing /system/xbin/su\n" +
				"adb shell /system/bin/busybox rm -f /system/xbin/su\n" +
				"echo Creating symlink to /system/bin/su in /system/xbin\n" +
				"adb shell /system/bin/busybox ln -s /system/bin/su /system/xbin/su\n" +
				"echo Now rebooting device ...\n" +
				"adb reboot\n" +
				":END\n" +
				"echo.\n" +
				"pause\n" +
				"exit");
				
		//Stream in Datei schreiben
		writer.flush();
								
		//Stream schliessen
		writer.close();
								
		Runtime.getRuntime().exec("cmd /C start " + System.getProperty("user.dir") + 
		"\\ADB\\wireless_mkroot.bat");
	}
	else{
		JOptionPane.showMessageDialog(null, "Bitte geben Sie eine gueltige IP-Adresse an", 
		"Fehler", JOptionPane.OK_OPTION);
	}
}
else
{
	Runtime.getRuntime().exec("cmd /C start " + System.getProperty("user.dir") + 
	"\\ADB\\mkroot.bat"); 
}	
	}
		catch (IOException e) {
			// TODO Auto-generated catch block
			e.printStackTrace();
		}
	}
}
\end{lstlisting}

Auf dem im Listing gezeigten Wege ist also nun möglich für jedes über Wlan angeschlossene Gerät eine entsprechende Datei zu erzeugen, die dann die angegebenen Befehle an das Gerät sendet. Somit ist es nun leichter möglich die Konfiguration des Odys Xelio zu verändern. Diese Vorgehensweise kann für jede andere Einstellung (Softwareinstallation oder die Aktivierung der Hardwarebuttons) genutzt werden. 

Jedoch hat diese Technik nicht den Weg in die Ausarbeitung des Projektes gefunden, da die gegebenen WLan-ADB-Apps nicht die Möglichkeit bieten die Verbindung mit einem Passwort zu sichern. Somit wäre es für jede Person möglich, welche die IP-Adresse des Gerätes besitzt, Kontrolle über das Tablet zu erlangen und Veränderungen daran vorzunehmen. Um dies zu verhindern muss also die eingesetzte WLan-ADB-App die Option bieten vor der Verbindung ein Passwort abzufragen. Eine weiteres Mittel um Wireless-ADB auch mit den bisher existierenden Wireless-ADB-Apps einzusetzen, wäre die Einrichtung eines eigenen Netzwerkes für die Tablets innerhalb des Gebietes in dem sie eingesetzt werden. So könnte die Anzahl an Personen, welche Zugriff auf die Tablets haben, stark begrenzt werden und es könnte somit auch vorher eine genaue Auswahl getroffen werden, welche Personen die Rechte haben Einstellungen zu verändern.

\section{Konzept zur Netzwerkanbindung}
Um die intelligenten Türschilder in der gesamten Fachhochschule bzw. erstmal im Fachbereich ETI zu realisieren, muss noch ein Lösung erarbeitet werden wie man die Türschilder an ein Netzwerk mit Samba Freigaben anbindet.\\

An dieser Stelle möchte wir ein mögliches Konzept dafür vorstellen. Eine erste Idee wäre es die Türschilder zum Beispiel in ein Campus weites bestehendes WLan mit einzubinden und dort Samba Shares zu erlauben. Da dann womöglich sehr viele Leute im gleichen Netzwerk sind, müsste man sich bei den späteren Freigaben für die Türschilder ziemlich sichere Passwörter verwenden, um Missbrauch vorzubeugen. Zu sichere Passwörter würden nach unserer Meinung aber der Komfortabilität entgegenwirken. Daher wäre es besser ein eigenes WLan einzurichten, was wiederum auch für Wireless-ADB von Vorteil wäre.\\

In diesem WLan könnte dann eine Virtuelle Maschine (VM) mit einem Samba Dienst eingehängt sein. Solch ein VM wurde von uns auch vorbereitet und wird auf einem Datenträger als OVA-(Open Virtualization Format)-Archiv mitgeliefert.\\

Diese vorbereitete VM enthält ein Debian Squeeze System mit installierten Samba Dienst. Der Dienst ist so konfiguriert das man einen Nutzer auf dem System anlegt z.B. als Nutzername ``raum319'', dann wird automatisch das Home-Verzeichnis des Nutzers, was den selben Namen als Verzeichnisnamen hat freigegeben.\\

Um einen korrekten Nutzer für das Linux System und den Samba Dienst zu erzeugen, wurde außerdem ein Skript erstellt, was unter dem Home-Verzeichnis des root-Nutzers liegt.\\

Dieses ruft man dann wie folgt auf:
\vspace{.3cm}
\begin{center}
\verb|./create_idc_samba_user %Nutzername% %Passwort%|
\end{center}
\vspace{.3cm}
Mithilfe einer zeilenweisen Liste von Nutzernamen und Passwörtern, könnte man dann auch relativ schnell mehrere Nutzer anlegen.\\

Daraufhin kann man die Synchronisation Applikation auf den Tablets wie im Abschnitt \ref{pcfilesync_subsec} \nameref{pcfilesync_subsec} konfigurieren.\\

Außerdem muss die VM von den PCs der Professoren bzw. von allen Leuten erreichbar sein, die ein solches Türschild mit neuen Daten bestücken wollen. Dazu muss dann nur die entsprechende Freigabe auf dem Rechner geöffnet werden und dort die Bilddateien gelöscht bzw. hinzugefügt werden. 

% Ich denke hier sollte wir/ich noch mal meine Samba VM erwähnen, wie ich mir dem Umgang mit dieser vorgestellt habe

\end{flushleft}
% Zu jeder Arbeit in unserem Bereich gehört eine Leistungsbewertung. Aus diesem Kapitel sollte hervorgehen, welche Methoden angewandt worden, die Leistungsfähigkeit zu bewerten und welche Ergebnisse dabei erzielt wurden. Wichtig ist es, dem Leser nicht nur ein paar Zahlen hinzustellen, sondern auch eine Diskussion der Ergebnisse vorzunehmen. Sehr gut ist, wenn man zunächst diskutiert und plausibel macht, welche Ergebnisse man erwartet, und dann eventuelle Abweichungen diskutiert.

    \include{Schlussfolgerung}    
    \chapter{ Zusammenfassung }
\vspace{-3cm}
\begin{flushleft}
Im Rahmen dieser Projektarbeit wurde mit Hilfe von Java, dem Betriebssystem ``Android'' und dafür entwickelte Software eine Möglichkeit entwickelt, welche ein Tablet-PC in einen digitalen Bilderrahmen verwandelt, um so den Aufwand der Pflege von Informationen zu verringern.
\newline

Im Einleitungskapitel Kapitel \ref{chap:Einleitung} wurde ein Überblick darüber gegeben, wie die aktuelle Situation in Bezug auf die Verteilung von Informationen an der Fachhochschule Stralsund aussieht und welche Veränderungen an diesem Zustand vorgenommen werden sollen. 
\newline

Daran anschließend wurde im Kapitel \ref{chap:Grundlagen} ``Grundlagen'' aufgezeigt, welche Technologien in der Arbeit genutzt werden, um die gestellte Aufgabe zu lösen. 
\newline

In Kapitel \ref{chap:Stand} ``Stand'' wurde noch einmal genauer beschrieben, wie die Türschilder bzw. Schaukästen genutzt werden und es wurde die Problemstellung noch einmal genauer definiert. Ebenfalls wurde eine bereits vorhandene Lösung für das Problem präsentiert und warum dieses nicht genutzt werden kann. 
\newline

Der nächste Abschnitt (Kapitel \ref{sec:Anforderungen}) gab eine Übersicht darüber, unter welchen Anforderungspunkten die Entwicklung des Projekts vollzogen werden sollte. Es erfolgte dafür eine Einteilung in funktionale und nichtfunktionale
Anforderungen. Dabei stützte sich die Erstellung der funktionalen Anforderungen auf die Formulierung von User Stories und die nichtfunktionalen Anforderungen auf den Standard ``ISO 9126''.
\newline

Das Kapitel \ref{chap:Entwurf} ``Entwurf'' setzte sich damit auseinander, welche Umsetzungsmöglichkeiten es für das vorliegende Problem gibt. Dabei entstanden zwei verschiedene Ansätze.
Als erstes wurde dabei Untersucht, ob ein digitaler Bilderrahmen als Lösung einsetzbar ist. Jedoch wurde während der Bearbeitung festgestellt, dass auf Grund einer neuer Firmwareversion des Bilderrahmens dieser nicht mehr für die Umsetzung geeignet ist. 
Es musste nun eine neue Option gesucht werden, um das Projekt weiterzuführen. Dabei fiel die Wahl auf einen Tablet-PC auf Android-Basis. Das Kapitel beschrieb daraufhin die Vorzüge des Tablet-PCs gegenüber dem Bilderrahmen und welche Arbeiten noch am Tablet durchgeführt werden müssen.
\newline

\newpage
Im Abschnitt \ref{chap:Umsetzung} wurde nun genauer darauf eingegangen, wie die im vorherigen Kapitel beschriebenen nötigen Arbeitsschritte umgesetzt wurden und wie die Versuche mit den digitalen Bilderrahmen aussahen.
\newline

Das Kapitel \ref{chap:Leistungsbewertung} zeigte, welche zuvor aufgezeigten Probleme gelöst werden konnten und an welchen Stellen noch Verbesserungen vorgenommen werden können.
\end{flushleft}

% Zu einer runden Arbeit gehört auch eine Zusammenfassung, die eigenständig einen kurzen Abriß der Arbeit gibt. Eine halbe bis ganze DINA4 Seite ist angemessen. Dafür läßt sich keine Gebrauchsanweisung geben (für irgendetwas müssen die Betreuer ja auch noch da sein). Jetzt sollen noch einige, eher generelle Hinweise gegeben werden:


    \chapter{Gebrauchsanweisung}
\vspace*{-3cm}
\begin{flushleft}
In diesem Kapitel werden die Funktionen erläutert, die die mitgelieferte Software bietet. Dabei erfolgt eine Unterteilung in die Erstellung von Notizen und die Wartung bzw. Inbetriebnahme des Xelio-Tablets. Weiterhin wird die für die Umsetzung des Projekts nötige Treiberinstallation erklärt. 

\section{Installieren des Treibers}
% TODO USB-Debugging aktivieren erwähnen

\section{Bedienung der Software am PC}
\subsection{Erstellung einer Notiz}
Um für das Tablet eine Nachricht zu erzeugen, muss die eigentliche Textnachricht in eine Bilddatei umgewandelt werden. Um diese Umwandlung zu vollziehen kann die mitgegebene Software genutzt werden. Dafür müssen folgende Schritte durchgeführt werden:
\begin{itemize}
  \item Klicken sie auf den Button ``Vorlage wählen''.
    \subitem Es öffnet sich nun ein Dialog für die Auswahl einer Bilddatei. Diese wird später der Hintergrund sein, auf dem sich die 			 Nachricht befindet. Bestätigen Sie ihre Auswahl mit ``öffnen''.
    \subitem Es wurde nun der Dateipfad zum Hintergrund in das entsprechende Textfeld eingetragen. 
    
  \item Im Feld ``Nachricht'' geben Sie nun die Nachricht ein, die auf dem Tablet erscheinen soll. 
    \subitem Hinweis: Es können bei der Eingabe selbstständig Zeilenumbrüche vorgenommen werden. Jedoch erfolgt seitens der Software noch 	  	 eine Kontrolle ob die eingegebene Nachricht für das Bild passend ist. Sollte dies nicht der Fall sein, werden weitere 	
	     Zeilenumbrüche eingefügt.  
	     
  \item Im nächsten Schritt kann nun mit Hilfe der Combobox die Schriftgröße der zu erscheinenden Nachricht eingestellt werden. 
  
  \item Sie können sich nun vor dem eigentlichen Erstellen der benötigten Bilddatei noch eine Vorschau der Notiz zeigen lassen. Dazu drücken   
        Sie auf den Button ``Vorschau anzeigen''. Es erscheint nun ein Fenster mit der Vorschau, welches Sie nach der Betrachtung wieder schließen können.
        
  \item Es können nun noch Veränderungen an der Nachricht vorgenommen werden. Entspricht die Notiz den Anforderungen kann mit einem Klick 	
	auf den Button ``Notiz erstellen'' ein Dialog zum Speichern der Bilddatei aufgerufen werden. Es muss nun ein Speicherort bestimmt werden und die Datei muss benannt werden. 
	
  \item Nun kann die Datei auf das Tablet übertragen werden und daraufhin auf dem Tablet angezeigt. 
\end{itemize}

Im nun folgenden Abschnitt wird der zweite Teil der Software genauer vorgestellt.

\subsection{Softwareinstallation und Wartung}
In diesem Absatz werden die Funktionen der vier verschiedenen Buttons im unteren Bereich der Software erklärt.

\begin{itemize}
  \item Button ``Xelio Rooten''
    \subitem Ist das Xelio an den PC angeschlossen und ist das USB-Debugging aktiviert, kann mit einem Klick auf den Button das Rooting des 	
	     Tablets starten. Während dieses Vorgangs kommt es zu einem Neustart des Geräts. Danach besitzt man Root-Zugang auf dem Xelio.
	     
  \item Button ``Software installieren''
    \subitem Nachdem das Tablet gerootet wurde, kann nun die Installation der nötigen Software erfolgen. Dies geschieht mit diesem Button. 	
	     Nach einem Klick erscheint ein Fenster welches den aktuellen Status ausgibt. Sollte sich das Fenster von alleine schließen, stoßen Sie den Vorgang erneut an. 
	 
  \item Buttons ``Hardwarebuttons aktivieren'' und ``Hardwarebuttons deaktivieren''
    \subitem Um das Tablet vor unerwünschten Eingaben zu schützen wurde die Möglichkeit geschaffen, die an der Vorderseite angebrachten 	
             Hardwarebuttons zu deaktivieren. Müssen nun allerdings doch Eingaben am Tablet selbst erfolgen, kann mit Hilfe der Software die Buttons wieder aktiviert werden. Für beide Fälle gilt, dass ein Neustart des Systems benötigt wird. Während der Ausführung des Befehls erscheint wieder ein Fenster, welches den aktuellen Status ausgibt. Sollte sich dieses wieder von alleine schließen, starten Sie den Vorgang bitte erneut.
\end{itemize}


\section{Bedienung der Verwaltungsapplikation auf dem Tablet}
  Nachdem das Tablet durch die entwickelte Verwaltungssoftware für den PC vorbereitet und außerdem ein entsprechender Samba-Server eingerichtet wurde, kann es nun auf dem Tablet weitergehen.
  \begin{itemize}
  \item{Konfigurieren des WLAN}
    Auf dem Tablet muss zunächst das WLAN konfiguriert werden.
    \begin{itemize}
      \item Klicken sie in der unteren rechte Ecke des Touchscreens auf die Uhr. Dort sollte sich ein Popup-Dialog öffnen.
      \item Dort klicken sie bitte auf das obere Feld mit der Uhr, daraufhin sollte sich der Dialog aktuallisieren.
      \item Hier kann man nun auf die Schaltfläche ``WLAN'' klicken, worauf sie in die WLAN-Einstellungen gelangen.
      \item Auf der rechten Seite sind nun die gefundenen WLAN-Netzwerke zusehen. Falls nicht, müssen sie auf der linken Seite den WLAN-Schalter auf ``AN'' stellen.
      \item Nachdem sie sich im WLAN mit vorhandenen Samba-Server angemeldet haben, können sie in der unteren linken Ecke auf die Zurück-Schaltfläche klicken.
    \end{itemize}
  \end{itemize}
  Die restlichen Einstellungen können nun über die entwickelte Verwaltungs-Applikation ``InteligentDoorplateControl'' vorgenommen werden.
  \subsection{Konfigurieren der Applikation - NoLock}
    Die NoLock-Applikation dient dazu, zu verhindern das das Tablet nach dem Bildschirm ausschalten den Lockscreen aktiviert.
    \begin{itemize}
      \item Starten sie die App mittels der Schaltfläche ``Configure NoLock!''.
      \item Falls dort noch ``Lock enabled'' steht klicken sie einmal dort drauf und schon sollte dort ``Lock disabled'' erscheinen, womit der Lockscreen abgeschaltet wurde.
      \item Falls dort schon ``Lock disabled'' zu sehen ist, ist der Lockscreen bereits deaktiviert. Das liegt daran das dies nur einmal gemacht werden muss, solange das Tablet nicht zurückgesetzt wird.
    \end{itemize}
  \subsection{Konfiguration der Applikation - ScreenOff}
    Die ScreenOff App ist ein wesentlicher Bestandteil des Standby-Service. Die Applikation wird automatisch ausgeführt wird, wenn die konfigurierte Sleeptime erreicht ist.
    Die Konfiguration von ScreenOff, ist eigentlich nur eine Aktivierung dieser. Die Applikation benötigt die freigabe als Geräteadministrator. Man muss hierbei die App nur zweimal starten und zwar über den Button ``Configure ScreenOff!''. Beim ersten mal werden einige Hinweise gegeben, diese kann man mit \textit{OK} bestätigen. Das zweite mal wird gefragt ob die Applikation Geräteadministrator werden darf, was man mit \textit{Aktivieren} bestätigen muss.
    Daraufhin ist die Applikation ordnungsgemäß aktiviert. Falls man die funktionalität prüfen will, kann  man einfach noch mal auf den entsprechenden Button drücken wodrauf das Tablet den Bildschirm ausschalten sollte. Solange die App nicht zurückgesetzt wird, bleibt die Aktivierung erhalten.
  \subsection{Konfigurieren der Applikation - PCFileSync}
  Die PCFileSync-Applikation dient dazu, ein bestimmtes Verzeichnis mit einer entsprechenden Netzwerkfreigabe, die in diesem Projekt über das SMB-Protokoll realisiert wird, zu synchronisieren.
    \begin{itemize}
      \item Starten sie die App mittels der Schaltfläche ``Start/configure PcFileSync!''.
      \item In der App rufen sie nun das Kontextmenü über die Hardware bzw. Software Options-Taste auf.
      \item Dort klicken sie nun auf ``Settings'' und im darauf folgenden Dialog ``Profiles settings''.
      \item Hier muss nun ein neues Profil angelegt werden, um die Synchronisation mit einem lokalen Verzeichnis und einem entsprechenden Netzwerkverzeichnis sicherzustellen. Dazu klicken sie lange in die freie Fläche der Activity, bis sich ein Dialog öffnet. Dort wählen sie ``Add SMB Profile''.
      \item Daraufhin gelangen sie zu einer Activity mit vielen Einstellungsmöglichkeiten. In der folgenden Abbildung ist zu sehen wie man diese Felder ausfüllen sollte, natürlich mit kleineren Anpassungen wie Raumnummer usw.
      \end{itemize}
      \begin{figure}[htb]
        \centering
        \includegraphics[width=.7\textwidth]{pcfilsync_conf.png}\\ % PNG-File
      \end{figure}
      Nun ist die App fertig konfiguriert und der Synchronisation-Service sollte schon automatisch gestartet sein.
      Ab dem zweiten Mal, muss die PcFileSync-App nur noch einmal gestartet werden, damit der Service startet, die Konfiguration bleibt erhalten.    
  \subsection{Konfigurieren der Applikation - PhotoFrameApp}
  	Die PhotoFrame-Applikation dient dazu, dass die Bilder aus dem synchronisierten Verzeichnis mittels einer Diashow ausgegeben werden.
    \begin{itemize}
      \item Starten sie die App mittels der Schaltfläche ``Start/configure PhotoFrameApp!''.
      \item Am oberen Rand sehen sie dann in der App eine Schaltfläche mit einem kleinen Zahnrad, klicken sie dort drauf, um in die Einstellungen zu gelangen.
      \item Angelangt im Einstellungs-Dialog, legt man zuerst das Intervall zwischen den Bildern fest. Zum Beispiel ein Intervall von 20 Sekunden.
      \item Als nächstes legt man die Art der Reihenfolge der Diashow fest. Im Fall eines Türschilds, würden wir die Reihenfolge nach Datum epmfehlen.
      \item Daraufhin muss das erneute einlesen des Ordners aktiviert werdenm, dazu machen sie bitte einen Haken bei ``Erneutes Laden einschalten''. Darunter legen sie das Intervall fest, wie oft der Ordner neu eingelesen wird. Wir würden ein Intervall von 30 Sekunden vorschlagen, das sollte ausreichend oft sein.
      \item Darauffolgend kann man eine Option aktivieren, die es veranlasst, dass nach dem ersten mal starten einer Diashow, die App bei nächsten Start wieder die Diashow im selben Ordner startet. Diese Option empfehlen wir ebenfalls, man aktiviert sie mit einem Haken bei Punkt ``Nach dem Neustart automatisch Fotos anzeigen.''.
      \item \dots
    \end{itemize}
  \subsection{Standby-Service konfigurieren}
    Der Standby-Service wurde dazu entwickelt, das Tablet zu bestimmten Zeiten aus dem Standby aufzuwecken bzw. in den Standbymodus zu versetzen.
    \begin{itemize}
      \item In der rechten Spalte der IDC-App haben wir zwei Felder zum eingeben der Wake-UP- und Sleep-Time. Diese müssen dem vorgegebenen Format entsprechen und müssen ausserdem mit einer führenden 0 beginnen falls die Stunden einstellig sind.
      \item Danach können sie über den Button ``Start Standby \& NoTouch Service'' den Service starten.
      \item Nach dem Start des Service haben sie noch ca. 5 Minuten bis der Touchscreen gesperrt wird. Danach ist das Tablet nur noch durch einen Neustart wieder normal zu verwenden.
    \end{itemize}
    In der verbleibenden Zeitspanne, muss nun endgültig die PhotoFrameApp gestartet und die Diashow ausgeführt werden. Daraufhin ist das Tablet nun so konfiguriert, das es sich mit dem gewählten Verzeichnis synchronisiert, es eine Diashow mit dessen Inhalt ausgibt, es nicht von Unbefugten benutzt werden kann und es sich zu bestimmten Zeiten in den Standbymodus versetzt und daraus auch wieder erwacht.
\end{flushleft}


\backmatter % ab hier keine Nummerierung mehr
    \listoffigures
    \bibliographystyle{geralpha}
    %% TODO testen ob nicht am anfang besser is
    % glossar
%stefan auskommentiert%    \printglossary[style=altlist,title=Glossar]
    % abkürzungen
    % TODO \deftranslation[to=German]{Acronyms}{Abkürzungsverzeichnis}
%stefan auskommentiert%    \printglossary[type=\acronymtype,style=long]
    % symbole
%stefan auskommentiert%    \printglossary[type=symbolslist,style=long]
    % TODO bib daten in ordner
    %\bibliography{./Bib/stud77}
    %\section*{Selbständigkeitserklärung}
Ich versichere, die von mir vorgelegte Arbeit selbständig verfasst zu haben. Alle
Stellen, die wörtlich oder sinngemäß aus veröffentlichten oder nicht veröffentlichten
Arbeiten anderer entnommen sind, habe ich als entnommen kenntlich gemacht.
Sämtliche Quellen und Hilfsmittel sind angegeben. Die Arbeit hat mit gleichem bzw.
in wesentlichen Teilen gleichem Inhalt noch keiner Prüfungsbehörde vorgelegen.
 

\end{document}

% --------------------------------------------------------------- END
