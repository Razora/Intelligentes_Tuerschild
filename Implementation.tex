\chapter{ Umsetzung }\label{chap:Umsetzung}
\vspace*{-3cm}
\begin{flushleft}
%Hier greift man einige wenige, interessante Gesichtspunkte der Implementierung heraus. Das Kapitel darf nicht mit Dokumentation oder gar Programmkommentaren verwechselt werden. Es kann vorkommen, dass sehr viele Gesichtspunkte aufgegriffen werden müssen, ist aber nicht sehr häufig. Zweck dieses Kapitels ist einerseits, glaubhaft zu machen, daß man es bei der Arbeit nicht mit einem "Papiertiger" sondern einem real existierenden System zu tun hat. Es ist sicherlich auch ein sehr wichtiger Text für jemanden, der die Arbeit später fortsetzt. Der dritte Gesichtspunkt dabei ist, einem Leser einen etwas tieferen Eindruck in die Technik zu geben, mit der man sich hier beschäftigt. Schöne Bespiele sind "War Stories", also Dinge mit denen man besonders zu kämpfen hatte, oder eine konkrete, beispielhafte Verfeinerung einer der in Kapitel 3 vorgestellten Ideen. Auch hier gilt, mehr als 20 Seiten liest keiner, aber das ist hierbei nicht so schlimm, weil man die Lektüre ja einfach abbrechen kann, ohne den Faden zu verlieren. Vollständige Quellprogramme haben in einer Arbeit nichts zu suchen, auch nicht im Anhang, sondern gehören auf Rechner, auf denen man sie sich ansehen kann.
% hier könnte drin stehen:
% Versuch mit dem PHP-Skript für Kodak-Pulse und Router (DNS-MASQ, SSL-Strip)
% Entwicklung der Anwendersoftware und Androidapp (welche Funktionen brauch die app, wie eingepflegt, energieprobleme?)
\section{Versuche mit dem Kodak-Pulse Bilderrahmen}
  Trotz dessen das Projekt letztendlich nicht mit den Kodak-Pulse Bilderrahmen umgesetzt wurde, sollen hier die wichtigsten Schritte bei dem Versuch ihn zu verwenden erläutern werden, da es ein wesentlicher Bestandteil des Projektes war.
  \subsection{Anfragen des Kodak-Pulse Bilderrahmen umleiten}
    Als erste Idee, sollte ein lokaler Webserver auf die Domain \textbf{device.pulse.kodak.com} reagieren, wo drauf dann, das im Abschnitt \ref{subsec:nutzungBilderrahmen} \nameref{subsec:nutzungBilderrahmen} erwähnte PHP-Skript liegt. Dies sollte dann die korrekten Antworten an den Bilderrahmen zurückliefern.
    Dazu benötigten wir einen Router inkl. \textit{Dnsmasq}\footnote{Dnsmasq ist ein sehr einfacher, kombinierter DNS- und DHCP-Server für kleine bis mittlere Netzwerke, der eine leicht verständliche Konfiguration ermöglicht, und sehr zuverlässig arbeitet. Mithilfe dieses Tools ist es möglich einen gewählten FQDN auf eine vom Anwender selbst gewählte IP-Adresse zuleiten.}, einen Rechner der als Webserver fungiert und natürlich den Bilderrahmen.
    \begin{enumerate}
      \item Zuallererst müssen der Bilderrahmen und der Rechner so konfiguriert werden, das sie eine Netzwerkverbindung mit dem Router herstellen können. Dabei ist zu beachten, dass der Rahmen und der Rechner als DNS-Server und Gateway die IP-Adresse des Routers eingetragen haben.
      \item Auf dem Rechner, mit dem Betriebssystem Ubuntu 11.10, wurde ein Apache-Webserver installiert. Die Virtualhost-Konfiguration wurde so angepasst, dass das PHP-Skript für alle Anfragen an den Webserver aufgerufen wird. Dies wurde in der Standard-Virtualhost-Konfiguration mithilfe von einer \textit{RewriteRule} gemacht, diese leitet nun alle Anfragen entsprechend des RegEx an das PHP-Skript weiter. Außerdem muss der Apache auf Port 80 und Port 443 hören, weil der Rahmen einiges über das SSL-Protokoll verschlüsselt senden will. Der Port 80 ist der Standard HTTP-Port, auf diesen hört der Apache schon standardmäßig. Außerdem muss er nun auf Port 443 hören der Standard HTTPS-Port, dazu einfach die vorhandene Virtualhost-Konfiguration kopieren und an das Ende der bestehenden Datei anhängen. Nun muss nur noch der Port 80 mit Port 443 ersetzt werden.\\\vspace{.3cm}
      \textbf{RewriteRule:}
      \verb|RewriteRule /DeviceRest.* /kodak-pulse-picture-frame-server.php|\\
      \vspace{.5cm}
      \textbf{Port 80/443:}\\
      \verb|<VirtualHost *:80> ... </VirtualHost>|
      \verb|<VirtualHost *:443> ... </VirtualHost>|
      \item Nun kommunizieren der Rahmen und der Rechner schon mit dem Router. Außerdem ist der Webserver soweit korrekt konfiguriert. Was noch fehlt ist das der Webserver auf die FQDN \textbf{device.pulse.kodak.com} reagiert, da der Bilderrahmen alle seine Anfragen an diese Domain übermitteln will. Daher muss nun auf dem verwendeten Router das Tool \textit{Dnsmasq} benutzt werden. Hat man dieses eingeschaltet sucht der Dienst in der Systemdatei \textit{/etc/hosts} nach Einträgen, dort können wir nun angeben das bei Aufruf der Domain \textbf{device.pulse.kodak.com} die Pakete an die IP-Adresse des Webserver weitergeleitet werden.
    \end{enumerate}
    Damit sollte der Bilderrahmen sich nun eigentlich am angelegten Webserver Authentifizieren können und man somit Bilder auf den Bilderahmen bringen kann. Da die Firmware auf dem verwendeten Bilderrahmen, aber nun doch den Hostnamen mit dem SSL-Zertifikat vergleichen will klappt dies leider nicht. Da 1. der Hostnamen nicht mit dem SSL-Zertifikat übereinstimmen kann und 2. hat der Server garkein Zertifikat hinterlegt. Es wurde dann auch versucht verschiedene SSL-Zertifikate zu hinterlegen, was aber nicht den gewünschten Erfolg brachte. Das führte zu einer weiteren Überlegung ein Tool namens SSL-Strip zu nutzen.
  \subsection{Umgeleitete Anfragen durch das Tool SSL-Strip schicken}

\section{Beschreibung der Software zur Notizenerstellung und der Tabletverwaltung am PC}
Um den Nutzer später die Möglichkeit zu geben kleine Notizen zu erstellen und diese dann auf das elektronische Türschild zu übertragen, wurde eine Software auf Java-Basis geschaffen, die diese Funktionalität bietet. Weiterhin kann das Tablet damit auch auf den Einsatz als Türschild vorbereitet werden, das heißt es kann gerootet werden, die nötige Software kann darüber installiert werden und das aktivieren bzw. das deaktivieren der Hardwarebuttons ist damit möglich. 
Um nun also das Tablet für seinen späteren Einsatzzweck einzurichten, stellt die Software vier Buttons bereit. Diese werden nun im einzelnen genauer vorgestellt. 
Da ein nötiger Schritt das rooten des Gerätes ist, bietet die Software an, dies per Knopfdruck zu vollziehen. Wird also nun dieser Button betätigt, so wird im Hintergrund zunächst erkannt, auf welchem Betriebssystem das Programm ausgeführt wird. Dementsprechend wird dann entweder ein Batch-Skript (Windows) oder ein Bash-Skript (Linux) ausgeführt. Dabei wird eine ADB-Verbindung zum Tablet hergestellt und es werden auf dem Gerät die nötigen Einstellungen vorgenommen, z.B. das ersetzen von Systemdateien oder das ändern von Besitzrechten. Nach einem automatischen Neustart ist dann das Odys Xelio gerootet und es kann mit dem nächsten Schritt fortgefahren werden.

Dabei handelt es sich um das installieren aller nötigen Applikationen (kurz Apps) die für die Verwendung als Türschild nötig sind. Mit einem Klick auf den Button ``Software installieren'' beginnt der Installationsvorgang automatisch. Es wird wie schon beim rooten zuerst das ausführende Betriebssystem geprüft und danach entschieden, welches Batch- bzw. Bash-Skript gerufen wird. Es kommt dann zunächst zu einer Prüfung, ob alle nötigen Installationsprogramme (.apk-Dateien) vorhanden sind. Sollte dies nicht so sein, bricht der Vorgang ab. Ist die Prüfung jedoch erfolgreich abgeschlossen, wird eine ADB-Verbindung zum Gerät hergestellt und mit dem Befehl ``adb install *.apk'' die Software installiert (der Stern wird dabei mit dem Namen des Programms ersetzt). Das zuvor erschienene Konsolenfenster kann nun mit einem Tastendruck beendet werden und der Vorgang ist abgeschlossen. 

Im letzten Schritt müssen nun noch die Hardwarebuttons des Odys Xelio deaktiviert werden, damit keine Navigation mehr möglich ist. Auch dies ist mit der Software möglich. Der Vorgang wird mit dem Button ``Hardwarebuttons deaktivieren'' gestartet. Es wird erneut zuerst geprüft, ob alle nötigen Dateien vorliegen. Sollte dies so sein, wird wieder die ADB-Verbindung aufgebaut. Um nun die Hardwaretasten zu deaktivieren, wird die Datei ``sun4i-keyboard.kl'', welche sich in ``/system/usr/keylayout'' befindet, gegen eine veränderte Version ausgetauscht. Bei dieser Datei handelt es sich um eine Definition der Ereignisse die ausgelöst werden, sobald eine Taste gedrückt wird, zum Beispiel das Wechseln auf den Homescreen, das Zurückgehen oder das Öffnen des Menüs. Jedoch kann mit mit Hilfe des Ereignisses ``FOCUS'' eine Taste auch deaktiviert werden. Im Falle des Odys Xelio geschieht genau das. Alle Tasten bekommen dieses Ereignis zugewiesen und werden so deaktiviert. Um die Veränderungen abzuschließen, muss dann das Gerät neu gestartet werden. Um die Tasten dann wieder aktivieren zu können, muss in der Software der Button ``Hardwarebuttons aktivieren'' gedrückt werden. Es geschieht intern das selbe wie beim deaktivieren, jedoch wird wieder die originale Version der Datei ``sun4i-keyboard.kl'' eingefügt. 
Damit wäre das Tablet mit Hilfe des PCs konfiguriert. Alle weiteren Einstellungen müssen nun am Tablet selbst vorgenommen werden.

Die letzte Funktionalität der Software ist das erstellen von kleinen Notizen. Dafür kann der Nutzer eine von ihm im Vorhinein erstellte Vorlage für den Hintergrund wählen. Zu beachten ist dabei, dass die Mitte frei bleibt, denn dort wird der Text eingefügt. Ist diese ausgewählt, muss jetzt noch die Nachricht eingegeben und die Schriftgröße ausgewählt werden. Die Nachricht wird nun mit Hilfe der Methode ``drawString'' der Java-Klasse ``Graphics2d'' gerendert. Die Methode benötigt neben dem zu schreibenden String noch die Koordinaten in x- und y-Richtung. In der vorliegenden Software wird dabei immer der Bildmittelpunkt bestimmt. Dies geschieht mit Hilfe der Methoden ``getHeight()'' und ``getWidth()'' der ``Bufferedimage''-Klasse. Jedoch muss nur die y-Koordinate berechnet werden, da der Bildmittelpunkt in x-Richtung immer 0 entspricht. Für die y-Koordinate wird zunächst der Bildmittelpunkt in y-Richtung mit ``buffImg.getHeight/2 bestimmt.
Davon wird allerdings noch die Höhe der Schrift abgezogen. Diese kann mit Hilfe von ''g2d.getFontMetrics().getHeight())`` bestimmt werden. Dabei handelt es sich bei ''g2d`` um ein Graphics2d-Objekt. Diesem wird mit der Methode ''setFont()`` ein ''Font``-Objekt übergeben, welches die Schriftart, den Schriftstil und die Schriftgröße enthält. Anhand dieser Parameter kann nun auch die y-Koordinate für den Mittelpunkt ermittelt werden und das Bild kann erstellt werden. 

%Ich würde jetzt hiernach noch deine App erklären und dann die Überleitung zum nächsten Abschnitt machen.
\end{flushleft}