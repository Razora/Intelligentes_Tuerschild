\chapter{ Implementation }
\vspace*{-3cm}
\begin{flushleft}
%Hier greift man einige wenige, interessante Gesichtspunkte der Implementierung heraus. Das Kapitel darf nicht mit Dokumentation oder gar Programmkommentaren verwechselt werden. Es kann vorkommen, dass sehr viele Gesichtspunkte aufgegriffen werden müssen, ist aber nicht sehr häufig. Zweck dieses Kapitels ist einerseits, glaubhaft zu machen, daß man es bei der Arbeit nicht mit einem "Papiertiger" sondern einem real existierenden System zu tun hat. Es ist sicherlich auch ein sehr wichtiger Text für jemanden, der die Arbeit später fortsetzt. Der dritte Gesichtspunkt dabei ist, einem Leser einen etwas tieferen Eindruck in die Technik zu geben, mit der man sich hier beschäftigt. Schöne Bespiele sind "War Stories", also Dinge mit denen man besonders zu kämpfen hatte, oder eine konkrete, beispielhafte Verfeinerung einer der in Kapitel 3 vorgestellten Ideen. Auch hier gilt, mehr als 20 Seiten liest keiner, aber das ist hierbei nicht so schlimm, weil man die Lektüre ja einfach abbrechen kann, ohne den Faden zu verlieren. Vollständige Quellprogramme haben in einer Arbeit nichts zu suchen, auch nicht im Anhang, sondern gehören auf Rechner, auf denen man sie sich ansehen kann.
% hier könnte drin stehen:
% Versuch mit dem PHP-Skript für Kodak-Pulse und Router (DNS-MASQ, SSL-Strip)
% Entwicklung der Anwendersoftware und Androidapp (welche Funktionen brauch die app, wie eingepflegt, energieprobleme?)
\section{Versuche mit dem Kodak-Pulse Bilderrahmen}
  Trotz dessen das Projekt letztendlich nicht mit den Kodak-Pulse Bilderrahmen umgesetzt wurde, sollen hier die wichtigsten Schritte bei dem Versuch ihn zu verwenden erläutern werden, da es ein wesentlicher Bestandteil des Projektes war.
  \subsection{Anfragen des Kodak-Pulse Bilderrahmen umleiten}
    Als erste Idee, sollte ein lokaler Webserver auf die Domain \textbf{device.pulse.kodak.com} reagieren, wo drauf dann, das im Abschnitt \ref{subsec:nutzungBilderrahmen} \nameref{subsec:nutzungBilderrahmen} erwähnte PHP-Skript liegt. Dies sollte dann die korrekten Antworten an den Bilderrahmen zurückliefern.
    Dazu benötigten wir einen Router inkl. \textit{Dnsmasq}\footnote{Dnsmasq ist ein sehr einfacher, kombinierter DNS- und DHCP-Server für kleine bis mittlere Netzwerke, der eine leicht verständliche Konfiguration ermöglicht, und sehr zuverlässig arbeitet. Mithilfe dieses Tools ist es möglich einen gewählten FQDN auf eine vom Anwender selbst gewählte IP-Adresse zuleiten.}, einen Rechner der als Webserver fungiert und natürlich den Bilderrahmen.
    \begin{enumerate}
      \item Zuallererst müssen der Bilderrahmen und der Rechner so konfiguriert werden, das sie eine Netzwerkverbindung mit dem Router herstellen können. Dabei ist zu beachten, dass der Rahmen und der Rechner als DNS-Server und Gateway die IP-Adresse des Routers eingetragen haben.
      \item Auf dem Rechner, mit dem Betriebssystem Ubuntu 11.10, wurde ein Apache-Webserver installiert. Die Virtualhost-Konfiguration wurde so angepasst, dass das PHP-Skript für alle Anfragen an den Webserver aufgerufen wird. Dies wurde in der Standard-Virtualhost-Konfiguration mithilfe von einer \textit{RewriteRule} gemacht, diese leitet nun alle Anfragen entsprechend des RegEx an das PHP-Skript weiter. Außerdem muss der Apache auf Port 80 und Port 443 hören, weil der Rahmen einiges über das SSL-Protokoll verschlüsselt senden will. Der Port 80 ist der Standard HTTP-Port, auf diesen hört der Apache schon standardmäßig. Außerdem muss er nun auf Port 443 hören der Standard HTTPS-Port, dazu einfach die vorhandene Virtualhost-Konfiguration kopieren und an das Ende der bestehenden Datei anhängen. Nun muss nur noch der Port 80 mit Port 443 ersetzt werden.\\\vspace{.3cm}
      \textbf{RewriteRule:}
      \verb|RewriteRule /DeviceRest.* /kodak-pulse-picture-frame-server.php|\\
      \vspace{.5cm}
      \textbf{Port 80/443:}\\
      \verb|<VirtualHost *:80> ... </VirtualHost>|
      \verb|<VirtualHost *:443> ... </VirtualHost>|
      \item Nun kommunizieren der Rahmen und der Rechner schon mit dem Router. Außerdem ist der Webserver soweit korrekt konfiguriert. Was noch fehlt ist das der Webserver auf die FQDN \textbf{device.pulse.kodak.com} reagiert, da der Bilderrahmen alle seine Anfragen an diese Domain übermitteln will. Daher muss nun auf dem verwendeten Router das Tool \textit{Dnsmasq} benutzt werden. Hat man dieses eingeschaltet sucht der Dienst in der Systemdatei \textit{/etc/hosts} nach Einträgen, dort können wir nun angeben das bei Aufruf der Domain \textbf{device.pulse.kodak.com} die Pakete an die IP-Adresse des Webserver weitergeleitet werden.
    \end{enumerate}
    Damit sollte der Bilderrahmen sich nun eigentlich am angelegten Webserver Authentifizieren können und man somit Bilder auf den Bilderahmen bringen kann. Da die Firmware auf dem verwendeten Bilderrahmen, aber nun doch den Hostnamen mit dem SSL-Zertifikat vergleichen will klappt dies leider nicht. Da 1. der Hostnamen nicht mit dem SSL-Zertifikat übereinstimmen kann und 2. hat der Server garkein Zertifikat hinterlegt. Es wurde dann auch versucht verschiedene SSL-Zertifikate zu hinterlegen, was aber nicht den gewünschten Erfolg brachte. Das führte zu einer weiteren Überlegung ein Tool namens SSL-Strip zu nutzen.
  \subsection{Umgeleitete Anfragen durch das Tool SSL-Strip schicken}
    
\end{flushleft}