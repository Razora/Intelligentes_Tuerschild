\chapter{ Umsetzung }\label{chap:Umsetzung}
\vspace*{-3cm}
\begin{flushleft}
%Hier greift man einige wenige, interessante Gesichtspunkte der Implementierung heraus. Das Kapitel darf nicht mit Dokumentation oder gar Programmkommentaren verwechselt werden. Es kann vorkommen, dass sehr viele Gesichtspunkte aufgegriffen werden müssen, ist aber nicht sehr häufig. Zweck dieses Kapitels ist einerseits, glaubhaft zu machen, daß man es bei der Arbeit nicht mit einem "Papiertiger" sondern einem real existierenden System zu tun hat. Es ist sicherlich auch ein sehr wichtiger Text für jemanden, der die Arbeit später fortsetzt. Der dritte Gesichtspunkt dabei ist, einem Leser einen etwas tieferen Eindruck in die Technik zu geben, mit der man sich hier beschäftigt. Schöne Bespiele sind "War Stories", also Dinge mit denen man besonders zu kämpfen hatte, oder eine konkrete, beispielhafte Verfeinerung einer der in Kapitel 3 vorgestellten Ideen. Auch hier gilt, mehr als 20 Seiten liest keiner, aber das ist hierbei nicht so schlimm, weil man die Lektüre ja einfach abbrechen kann, ohne den Faden zu verlieren. Vollständige Quellprogramme haben in einer Arbeit nichts zu suchen, auch nicht im Anhang, sondern gehören auf Rechner, auf denen man sie sich ansehen kann.
% hier könnte drin stehen:
% Versuch mit dem PHP-Skript für Kodak-Pulse und Router (DNS-MASQ, SSL-Strip)
% Entwicklung der Anwendersoftware und Androidapp (welche Funktionen brauch die app, wie eingepflegt, energieprobleme?)
\section{Versuche mit dem Kodak-Pulse Bilderrahmen}
  Trotz dessen das Projekt letztendlich nicht mit den Kodak-Pulse Bilderrahmen umgesetzt wurde, sollen hier die wichtigsten Schritte bei dem Versuch ihn zu verwenden erläutert werden, da es ein wesentlicher Bestandteil des Projektes war.
  \subsection{Anfragen des Kodak-Pulse Bilderrahmen umleiten}
    Als erste Idee, sollte ein lokaler Webserver auf die Domain \textbf{device.pulse.kodak.com} reagieren, wo drauf, das im Abschnitt \ref{subsec:nutzungBilderrahmen} \nameref{subsec:nutzungBilderrahmen} erwähnte PHP-Skript liegt. Dies sollte die korrekten Antworten an den Bilderrahmen zurückliefern.
    Dazu benötigten wir einen Router inkl. \textit{Dnsmasq}\footnote{Dnsmasq ist ein sehr einfacher, kombinierter DNS- und DHCP-Server für kleine bis mittlere Netzwerke, der eine leicht verständliche Konfiguration ermöglicht, und sehr zuverlässig arbeitet. Mithilfe dieses Tools ist es möglich einen gewählten FQDN auf eine vom Anwender selbst gewählte IP-Adresse zuleiten.}, einen Rechner der als Webserver fungiert und natürlich den Bilderrahmen.
    \begin{enumerate}
      \item Zuallererst müssen der Bilderrahmen und der Rechner so konfiguriert werden, das sie eine Netzwerkverbindung mit dem Router herstellen können. Dabei ist zu beachten, dass der Rahmen und der Rechner als DNS-Server und Gateway die IP-Adresse des Routers eingetragen haben.
      \item Auf dem Rechner, mit dem Betriebssystem Ubuntu 11.10, wurde ein Apache-Webserver installiert. Die angezeigten Webseiten des Webservers werden über sogenannte Virtualhost-Konfiguration eingerichtet. Solch eine Konfiguration wurde so angepasst, dass das PHP-Skript, für alle Anfragen an den Webserver, aufgerufen wird. Dies wurde in der standardmäßig auf dem Webserver vorhandenen Virtualhost-Konfiguration mithilfe von einer \textit{RewriteRule} gemacht, diese leitet nun alle Anfragen entsprechend des regulären Ausdrucks an das PHP-Skript weiter. Außerdem muss der Apache nicht nur auf Port 80 und sondern auch auf Port 443 hören, weil der Rahmen einiges über das SSL-Protokoll verschlüsselt senden will. Der Port 80 ist der Standard HTTP-Port, auf diesen hört der Apache schon standardmäßig. Außerdem muss er nun auf Port 443 hören der Standard HTTPS-Port, dazu einfach die vorhandene Virtualhost-Konfiguration kopieren und an das Ende der bestehenden Datei anhängen. Nun muss nur noch der Port 80 mit Port 443 ersetzt werden.\\\vspace{.3cm}
      \textbf{RewriteRule:}
      \verb|RewriteRule /DeviceRest.* /kodak-pulse-picture-frame-server.php|\\
      \vspace{.5cm}
      \textbf{Port 80/443:}\\
      \verb|<VirtualHost *:80> ... </VirtualHost>|
      \verb|<VirtualHost *:443> ... </VirtualHost>|
      \item Nun kommunizieren der Rahmen und der Rechner schon mit dem Router. Außerdem ist der Webserver soweit korrekt konfiguriert. Was noch fehlt ist das der Webserver auf die FQDN \textbf{device.pulse.kodak.com} reagiert, da der Bilderrahmen alle seine Anfragen an diese Domain übermitteln will. Daher muss nun auf dem verwendeten Router das Tool \textit{Dnsmasq} benutzt werden. Hat man dieses eingeschaltet sucht der Dienst in der Systemdatei \textit{/etc/hosts} nach Einträgen, dort kann man angeben was für eine FQDN zu welcher IP-Adresse gehört. Es ist sozusagen ein statischer/manueller DNS-Server.\\\vspace{.3cm}
      \textbf{Beispieleintrag in /etc/hosts:}\\
      \verb|192.168.0.1 	device.pulse.kodak.com|
    \end{enumerate}
    Damit sollte der Bilderrahmen sich nun eigentlich am angelegten Webserver Authentifizieren können und man somit Bilder auf den Bilderrahmen bringen kann. Da die Firmware auf dem verwendeten Rahmen, aber nun doch den Hostnamen mit dem SSL-Zertifikat vergleichen will, klappt dies leider nicht. Da erstens der Hostnamen nicht mit dem SSL-Zertifikat übereinstimmen kann und zweitens hat der Server gar kein Zertifikat hinterlegt. Es wurde auch versucht verschiedene SSL-Zertifikate zu hinterlegen, was aber nicht den gewünschten Erfolg brachte. Das führte zu einer weiteren Überlegung ein Tool namens SSL-Strip zu nutzen.
  \subsection{Umgeleitete Anfragen durch das Tool SSL-Strip schicken}
    Um mit dem Tool SSL-Strip zu arbeiten ist die vorher beschriebene Konfiguration ebenfalls notwendig. Hinzu kommt nun in unserem Fall noch eine Virtuelle Maschine. Diese wurde mit einem vorgefertigten Image namens BackTrack erstellt. Dieses Linux-Image enthält viele Tools, um ein Netzwerk zu ``Sniffen''. Unter anderem enthält die VM auch das Tool SSL-Strip.
    
    Das Tool dient dazu einen empfangenen Netzwerkverkehr auf HTTPS Pakete zu durchsuchen und die mit HTTP zu ersetzen, somit ist es möglich z.B. GMAIL oder auch PAYPAl Anmeldedaten im Klartext mit zu lesen. Außerdem, was für unseren Einsatz wesentlich wichtiger ist, werden die Pakete so manipuliert weitergeschickt. Dadurch haben wir vermutet können wir den Bilderrahmen austricksen, was aber letztendlich auch nicht zum Erfolg führte.
    
    Die VM muss nun natürlich erstmal so konfiguriert werden das sie im gleichen Adressbereich wie der Router und Bilderrahmen hängt. Daraufhin muss nun die VM darauf vorbereitet werden den IPv4 Netzwerkverkehr komplett zu empfangen und auch wieder weiterzuleiten (IPv4-Forwarding).\par Wie man vllt. erkennen kann, versuchen wir jetzt einen Man-in-the-Middle Angriff durchzuführen, um somit die SSL-Verschlüsselung des Rahmens zu umgehen.\vspace{.3cm}
    
    \underline{Vorgehen:}\\
    \begin{itemize}
    \item Als erstes müssen wir IPv4-Forwarding aktivieren. Das geht unter Linux sehr einfach, in der Datei ``/proc/sys/net/ipv4/ip\_forward'' steht im normalfall eine ``0'' d.h. das IPv4-Forwarding deaktiviert ist. An diese Stelle muss nun eine ``1'' um das Forwarding zu aktivieren. Dazu schreibt man entweder mit einem Rootfähigen Editor eine ``1'' in die Datei bzw. führt mit Rootrechten in einem Terminal folgenden Befehl aus, der ebenfalls die ``1'' in die Datei schreibt.\\\vspace{.3cm}
    \begin{center}
      \verb+echo 1 > /proc/sys/net/ipv4/ip_forward+
    \end{center}    
    \item Daraufhin muss man sicherstellen, dass die Pakete vom Bilderrahmen an den Webserver auch den Umweg über die VM gehen. Dazu nutzen wir das Tool ``ARPSPOOF'', dieses ermöglicht es manipulierte ARP-Nachrichten zu verschicken und somit eine IP-Adresse eine andere MAC-Adresse zuzuweisen. Es muss nun eine Nachricht gesendet werden die dem Bilderrahmen bzw. dem Webserver vortäuscht das die Gateway IP des Netzes durch das die Pakete um jeden Preis durch wollen, die MAC-Adresse der VM mit SSL-Strip hat. Dadurch geht der Verkehr nun durch die VM. Folgend ein beispielhafter Aufruf:
    \begin{center}
      \verb+arpspoof -i eth0 -t 192.168.1.6 192.168.1.1+
    \end{center}
    \item Wenn man SSL-Strip startet hört es standardmäßig auf Port 10000, der dort antreffende HTTP/HTTPS Verkehr wird daraufhin manipuliert. Dazu benötigen wir also ein Route die den Verkehr auf Port 80 auf den Port 10000 umleitet. Dazu wird ein weiteres Tool IPTables genutzt, hiermit lassen sich sehr schnell komplexe Netzwerkrouten erstellen. Um nun das hier gewünschte Verhalten zu erreichen müssen wir folgenden Befehl ausführen.
    \begin{center}
      \verb+iptables -t nat -A PREROUTING -p tcp \+
      \verb+--destination-port 80 -j REDIRECT --to-ports 10000+
    \end{center}
    \item Nun ist die VM fertig vorbereitet, somit kann man zum Abschluss nun das SSL-Strip Tool starten. Man kann nun im laufenden Betrieb in den Logdaten sehen das der Verkehr manipuliert wird. Das Tool starten gelingt mit folgenden Befehl:
    \begin{center}
      \verb+python sslstrip.py+
    \end{center}
    \end{itemize}
    Somit wird nun der Netzwerkverkehr von HTTPS Anfragen gesäubert, was zu dem Erfolg führen sollte, dass der Rahmen nun das Zertifikat nicht prüfen will. Die Firmware des Rahmens lies sich, aber auch auf diese Weise nicht austricksen.
    
    Dadurch haben wir aus schon erwähnten Gründen gegen die Bilderrahmen entschieden.

\section{Beschreibung der Software zur Notizenerstellung und der Tabletverwaltung am PC}
Um den Nutzer später die Möglichkeit zu geben kleine Notizen zu erstellen und diese dann auf das elektronische Türschild zu übertragen, wurde eine Software auf Java-Basis geschaffen, die diese Funktionalität bietet. Weiterhin kann das Tablet damit auch auf den Einsatz als Türschild vorbereitet werden, das heißt es kann gerootet werden, die nötige Software kann darüber installiert werden und das aktivieren bzw. das deaktivieren der Hardwarebuttons ist damit möglich. 
Um nun also das Tablet für seinen späteren Einsatzzweck einzurichten, stellt die Software vier Buttons bereit. Diese werden nun im einzelnen genauer vorgestellt. 
Da ein nötiger Schritt das rooten des Gerätes ist, bietet die Software an, dies per Knopfdruck zu vollziehen. Wird also nun dieser Button betätigt, so wird im Hintergrund zunächst erkannt, auf welchem Betriebssystem das Programm ausgeführt wird. Dementsprechend wird dann entweder ein Batch-Skript (Windows) oder ein Bash-Skript (Linux) ausgeführt. Dabei wird eine ADB-Verbindung zum Tablet hergestellt und es werden auf dem Gerät die nötigen Einstellungen vorgenommen, z.B. das ersetzen von Systemdateien oder das ändern von Besitzrechten. Nach einem automatischen Neustart ist dann das Odys Xelio gerootet und es kann mit dem nächsten Schritt fortgefahren werden.

Dabei handelt es sich um das installieren aller nötigen Applikationen (kurz Apps) die für die Verwendung als Türschild nötig sind. Mit einem Klick auf den Button ``Software installieren'' beginnt der Installationsvorgang automatisch. Es wird wie schon beim rooten zuerst das ausführende Betriebssystem geprüft und danach entschieden, welches Batch- bzw. Bash-Skript gerufen wird. Es kommt dann zunächst zu einer Prüfung, ob alle nötigen Installationsprogramme (.apk-Dateien) vorhanden sind. Sollte dies nicht so sein, bricht der Vorgang ab. Ist die Prüfung jedoch erfolgreich abgeschlossen, wird eine ADB-Verbindung zum Gerät hergestellt und mit dem Befehl ``adb install *.apk'' die Software installiert (der Stern wird dabei mit dem Namen des Programms ersetzt). Das zuvor erschienene Konsolenfenster kann nun mit einem Tastendruck beendet werden und der Vorgang ist abgeschlossen. 

Im letzten Schritt müssen nun noch die Hardwarebuttons des Odys Xelio deaktiviert werden, damit keine Navigation mehr möglich ist. Auch dies ist mit der Software möglich. Der Vorgang wird mit dem Button ``Hardwarebuttons deaktivieren'' gestartet. Es wird erneut zuerst geprüft, ob alle nötigen Dateien vorliegen. Sollte dies so sein, wird wieder die ADB-Verbindung aufgebaut. Um nun die Hardwaretasten zu deaktivieren, wird die Datei ``sun4i-keyboard.kl'', welche sich in ``/system/usr/keylayout'' befindet, gegen eine veränderte Version ausgetauscht. Bei dieser Datei handelt es sich um eine Definition der Ereignisse die ausgelöst werden, sobald eine Taste gedrückt wird, zum Beispiel das Wechseln auf den Homescreen, das Zurückgehen oder das Öffnen des Menüs. Jedoch kann mit mit Hilfe des Ereignisses ``FOCUS'' eine Taste auch deaktiviert werden. Im Falle des Odys Xelio geschieht genau das. Alle Tasten bekommen dieses Ereignis zugewiesen und werden so deaktiviert. Um die Veränderungen abzuschließen, muss dann das Gerät neu gestartet werden. Um die Tasten dann wieder aktivieren zu können, muss in der Software der Button ``Hardwarebuttons aktivieren'' gedrückt werden. Es geschieht intern das selbe wie beim deaktivieren, jedoch wird wieder die originale Version der Datei ``sun4i-keyboard.kl'' eingefügt. 
Damit wäre das Tablet mit Hilfe des PCs konfiguriert. Alle weiteren Einstellungen müssen nun am Tablet selbst vorgenommen werden.

Die letzte Funktionalität der Software ist das erstellen von kleinen Notizen. Dafür kann der Nutzer eine von ihm im Vorhinein erstellte Vorlage für den Hintergrund wählen. Zu beachten ist dabei, dass die Mitte frei bleibt, denn dort wird der Text eingefügt. Ist diese ausgewählt, muss jetzt noch die Nachricht eingegeben und die Schriftgröße ausgewählt werden. Die Nachricht wird nun mit Hilfe der Methode ``drawString'' der Java-Klasse ``Graphics2d'' gerendert. Die Methode benötigt neben dem zu schreibenden String noch die Koordinaten in x- und y-Richtung. In der vorliegenden Software wird dabei immer der Bildmittelpunkt bestimmt. Dies geschieht mit Hilfe der Methoden ``getHeight()'' und ``getWidth()'' der ``Bufferedimage''-Klasse. Jedoch muss nur die y-Koordinate berechnet werden, da der Bildmittelpunkt in x-Richtung immer 0 entspricht. Für die y-Koordinate wird zunächst der Bildmittelpunkt in y-Richtung mit ``buffImg.getHeight/2 bestimmt.
Davon wird allerdings noch die Höhe der Schrift abgezogen. Diese kann mit Hilfe von ''g2d.getFontMetrics().getHeight())`` bestimmt werden. Dabei handelt es sich bei ''g2d`` um ein Graphics2d-Objekt. Diesem wird mit der Methode ''setFont()`` ein ''Font``-Objekt übergeben, welches die Schriftart, den Schriftstil und die Schriftgröße enthält. Anhand dieser Parameter kann nun auch die y-Koordinate für den Mittelpunkt ermittelt werden und das Bild kann erstellt werden.

\section{Beschreibung der IntelligentDoorplateControl Applikation zur Einrichtung des Tablets}
Diese Applikation soll zum ersten die Einrichtung als Türschild direkt auf dem Tablet erleichtern. Dazu lassen sich mittels der Applikation die anderen Apps starten die zu Nutzung als Türschild erforderlich sind. Des weiteren sperrt die Applikation nach gewisser Zeit den Touchscreen, um die Nutzung Unbefugter vorzubeugen. Außerdem wird mit ihrer Hilfe die Statusbar versteckt um das volle Display nutzen zu können. Und die Applikation trägt auch ihren Beitrag zur Green-IT bei, in dem sie das Tablet in einem bestimmten Zeitraum in einen Standbymodus versetzt.

Die Applikation ist im Quellcode dabei also eher recht simpel gehalten, da sie überwiegend nur als eine Art Oberfläche dient. Dabei gibt es eine Activity die alle Schaltflächen der App beinhaltet, außerdem gibt es noch eine Klasse namens ``Device'' diese wurde dem Projekt \textbf{HideBar}\footnote{Dieses Projekt ist selbst eine Android-Applikation die es ermöglicht die Statusbar zu manipulieren. Der Quellcode liegt offen auf Github, da für uns nur das verstecken der Leiste entscheidend war, haben wir anstatt die Applikation selbst zu nutzen, nur eine Klasse des Projekt in Verwendung.\url{http://ppareit.github.com/HideBar/}} entnommen und dient zum verstecken der Statusbar des Tablets. Um den Standbymodus und das blockieren des Touchscreens Zeitgesteuert zu verwalten, gibt es noch einen Service der eine BroadcastReceiver jede Minute aufruft. Wobei im Service ein partieller Wakelock aktiviert wird, damit das Tablet nie in den Tiefschlaf fällt. Im Receiver wird nach den ersten 5 Minuten der Touchscreen blockiert und ausserdem je nach Angaben des Zeitraums für den Standbymodus, das Tablet mit Hilfe der App ``ScreenOffAndLock'' in den Standbybetrieb versetzt und mithilfe eines FullWakelocks in normal betrieb versetzt.

Nun möchte ich noch mal genauer eine Code Abschnitte erläutern. Und zwar einmal allgemein das Starten der genutzten Applikation aus einer Activity und zweitens möchte ich ein wenig genauer erläutern wir der Standby \& NoTouch Service funktioniert.

Nun zum Starten einer Applikation aus einer Activity. Zuerst ist es sehr Sinnvoll zu prüfen ob die Applikation die man Starten will auch auf dem Android-Gerät vorhanden ist. Dazu benötigt man von der Applikation den \textit{PackageName} und außerdem den Pfad der Main-Activity der zu startenden Applikation.

Um die Prüfung nun durchzuführen habe ich mir eine kleine Funktion erstellt, die diese Überprüfung nun machen soll. Diese holt sich als erstes mittels des Paketnamen alle unter diesem Namen vorhandenen Activities und speichert sie in einem Array vom Typ \textbf{ActivityInfo}. Die einzelnen Activities werden dann mit der zu suchenden Activity verglichen. Falls sie gefunden wird, wird abgebrochen und ein TRUE zurückgegeben, ansonsten ein FALSE. Was dann einen Hinweis auslöst, der aussagt das die App nicht vorhanden ist.

\begin{lstlisting}[caption={Application Available Function}\label{AppAvailFunction},captionpos=t,language=JAVA] 
	static boolean isIntentAvailable(Context context, String pkgName, 
	String activity) {
		ActivityInfo[] activityInfo = null;
		try {
			activityInfo = context.getPackageManager().getPackageInfo(pkgName,
			PackageManager.GET_ACTIVITIES).activities;
		} catch (NameNotFoundException e) {
			e.printStackTrace();
		}
				
		if(activityInfo!=null) {
            for(int i=0; i < activityInfo.length; i++) {
            	if (activityInfo[i].name.compareTo(activity) == 0) {
            		return true;
            	}
            }
        }
		return false;
	}
\end{lstlisting}

Ein Beispielhafter Aufruf der Funktion wäre folgender:

\begin{lstlisting}[caption={Application Available Function Call}\label{AppAvailFunctionCall}, captionpos=t, language=JAVA] 
final boolean available = MainActivity.isIntentAvailable(context,
			"com.katecca.screenofflock", "com.katecca.screenofflock.MainHelper");
\end{lstlisting}

Nun möchte ich zu einer näheren Erläuterung des Standby \& NoTouch Service kommen. Die Funktionalität die dieser Service bereitstellen soll ist einmal das Ein-/Ausschalten des Bildschirms zu vorher festgelegten Zeiten und außerdem soll der Service den Touchscreen sperren.

%Ich würde jetzt hiernach noch deine App erklären und dann die Überleitung zum nächsten Abschnitt machen.
\end{flushleft}