\chapter{Anforderungsdefinition} % (fold)
\label{sec:Anforderungen}
\vspace{-3cm}
\begin{flushleft}
In diesem Kapitel wird erläutert welche Ansprüchean die Soft- und Hardware gestellt werden.
Dazu werden die funktionalen Anforderungen in Form von ``User Stories'' beschrieben und die nichtfunktionalen Anforderungen werden durch ISO/IEC 9126 beschrieben.
% Dazu werden die funktionalen Anforderungen durch ``Use cases'' beschrieben und 
% die nichtfunktionalen Anforderungen werden durch \gls{ISOSQ} beschrieben.


\section{Funktionale Anforderungen} % (fold)
\label{sec:Funktionale Anforderungen}

\subsection{User Stories}
\label{sec:Usecases}
\begin{flushleft}
% eine einleitung wenn hier viel text folgt.
In diesem Abschnitt soll die geplante Verwendung der Soft- und Hardware erklärt werden.
Dafür werden die verschiedenen User Stories vorgestellt.

% warum kommt es zum einsatz ?
% Eine Softwarelösung für das Problem der Zahnbewegung soll hauptsächlich für zwei Dinge erfolgen.
% Zum einen soll die Software für die Lehre benutzt werden und zum anderen um Voraussagen für konkrete Behandlungen zu treffen.

% Die ``Use cases'' für diese Softwareentwicklung sind im Appendix Abschnitt \ref{sec:Use cases} auf Seite \pageref{sec:Use cases} zu finden, 
% werden hier aus Übersichtlichkeisgründen aber kurz erklärt.

\subsubsection{Dokumente auf das Türschild übertragen} % (fold)
\label{ssub:Daten auf das Türschild übertragen}
Als User möchte ich Dokumente auf das Türschild übertragen, um diese zu publizieren. 

\subsubsection{Dokumente vom Türschild entfernen}
\label{ssub:Dokumente vom Speicher des Türschilds entfernen}
Als User möchte ich Dokumente vom Türschild löschen, um veraltete Informationen nicht mehr zu verbreiten.  

\subsubsection{Das Türschild gegen Eingaben von Dritten schützen}
\label{ssub:Das Türschild gegen Eingaben von Dritten schützen}
Als Türschildsverwalter möchte ich das Türschild so einstellen, dass fremde Personen keine Veränderungen daran vornehmen können. 
\newline
\end{flushleft}



% Der Benutzer der Software soll in der Lage sein präparierte Inventor-Daten zu Laden.
% %% TODO erklärung des dateiformats ?
% Der Aufbau des Dateiformats wird im Anhang Abschnitte \ref{sec:Dateiformat} erklärt.
% Hierbei soll es dem Nutzer möglich sein die Datei durch einen Dateiauswahldialog auszuwählen um sie dann in das Programm zu laden.
% Sind die Daten geladen zeigt sich dem Nutzer eine Visuelle repräsentation der Zähne.





%TODO was einfügen
% section Funktionale Anforderungen (end)

\section{Nichtfunktionale Anforderungen} % (fold)
\label{sec:Nichtfunktionale Anforderungen}

\subsection{Benutzbarkeit} % (fold)
\label{sub:Benutzbarkeit}

Die Software muss einfach verständlich sein ( ``Verständlichkeit'' ) und sich einfach Erlernen lassen (``Erlernbarkeit'').
Eine einfache Benutzung steht dabei im Vordergrund ( ``Benutzbarkeit'' ), hingegen soll ``Attraktivität'' nur soweit gewährleistet sein,
dass sie die ``Verständlichkeit'' positiv beeinflusst. 

% subsection Benutzbarkeit (end)


\subsection{Wartbarkeit} % (fold)
\label{sub:Wartbarkeit}

Es soll die Möglichkeit bestehen, leicht neue Funktionen zur Software hinzuzufügen ohne den konzeptionellen Aufbau des Produkts zu verändern. 
% subsection Wartbarkeit (end)

\subsection{Übertragbarkeit} % (fold)
\label{sub:Übertragbarkeit}

Auf die Qualitätsanforderung ``Übertragbarkeit'' soll in dem Sinne eingegangen werden, dass die Software auf Linux und Windows lauffähig ist ( ``Anpassbarkeit'' ).

% subsection Übertragbarkeit (end)
% section Nichtfunktionale Anforderungen (end)

% chapter Anforderungsdefinition (end)
 
\end{flushleft}

