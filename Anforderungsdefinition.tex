\chapter{Anforderungsdefinition} % (fold)
\label{sec:Anforderungen}
\vspace{-3cm}
\begin{flushleft}
In diesem Kapitel wird erläutert welche Ansprüchean die Soft- und Hardware gestellt werden.
Dazu werden die funktionalen Anforderungen in Form von ``User Stories'' beschrieben und die nichtfunktionalen Anforderungen werden durch ISO/IEC 9126 beschrieben.
% Dazu werden die funktionalen Anforderungen durch ``Use cases'' beschrieben und 
% die nichtfunktionalen Anforderungen werden durch \gls{ISOSQ} beschrieben.


\section{Funktionale Anforderungen} % (fold)
\label{sec:Funktionale Anforderungen}

\subsection{User Stories}
\label{sec:Usecases}
\begin{flushleft}
% eine einleitung wenn hier viel text folgt.
In diesem Abschnitt soll die geplante Verwendung der Software erklärt werden.
Dafür werden die verschiedenen User Stories vorgestellt.

% warum kommt es zum einsatz ?
% Eine Softwarelösung für das Problem der Zahnbewegung soll hauptsächlich für zwei Dinge erfolgen.
% Zum einen soll die Software für die Lehre benutzt werden und zum anderen um Voraussagen für konkrete Behandlungen zu treffen.

% Die ``Use cases'' für diese Softwareentwicklung sind im Appendix Abschnitt \ref{sec:Use cases} auf Seite \pageref{sec:Use cases} zu finden, 
% werden hier aus Übersichtlichkeisgründen aber kurz erklärt.

\subsubsection{Daten aus HIS-LSF auslesen} % (fold)
\label{ssub:Daten aus HIS-LSF auslesen}
% Es sollen aus den Datenbeständen von \acs{HIS}-\acs{LSF} Informationen ausgelesen werden, die dann weiter verarbeitet werden und mit \acs{Moodle} abgeglichen werden. 
Als User möchte ich Informationen aus den Datenbeständen von \acs{HIS}-\acs{LSF} auslesen, um diese an die Datenverarbeitung zu übertragen.

\subsubsection{Daten aus Moodle auslesen}
\label{ssub:Daten aus Moodle auslesen}
% Es sollen aus den Datenbeständen von \acs{Moodle} Informationen ausgelesen werden, die dann weiter verarbeitet werden und mit \acs{HIS}-\acs{LSF} abgeglichen werden. 
Als User möchte ich Informationen aus den Datenbeständen von \acs{Moodle} auslesen, um zu kontrollieren, ob schon ein User in der Datenbank von \acs{Moodle} vorhanden ist.

\subsubsection{Daten in Moodle schreiben}
\label{ssub:Daten in Moodle schreiben}
% Die Software soll aus dem \acs{HIS}-\acs{LSF} übermittelte Daten an die Geschäftslogik von \acs{Moodle} senden, damit diese die Informationen in die eigene Datenbank aufnehmen kann.
Als User möchte ich Daten an die Geschäftslogik von \acs{Moodle} senden, um einen Nutzer in die Datenbank aufzunehmen. 

\subsubsection{Kontrolle der Richtigkeit und Vollständigkeit der Daten}
\label{ssub:Kontrolle der Richtigkeit der Daten}
% Die Schnittstelle muss dazu in der Lage sein, während der Übermittlung die Daten auf Richtigkeit und Vollständigkeit zu prüfen. 
Als User möchte ich, dass vor der Übermittlung von Daten, diese kontrolliert werden, damit keine falschen Informationen in der Datenbank vorhanden sind.

\end{flushleft}



% Der Benutzer der Software soll in der Lage sein präparierte Inventor-Daten zu Laden.
% %% TODO erklärung des dateiformats ?
% Der Aufbau des Dateiformats wird im Anhang Abschnitte \ref{sec:Dateiformat} erklärt.
% Hierbei soll es dem Nutzer möglich sein die Datei durch einen Dateiauswahldialog auszuwählen um sie dann in das Programm zu laden.
% Sind die Daten geladen zeigt sich dem Nutzer eine Visuelle repräsentation der Zähne.





%TODO was einfügen
% section Funktionale Anforderungen (end)

\section{Nichtfunktionale Anforderungen} % (fold)
\label{sec:Nichtfunktionale Anforderungen}

\subsection{Benutzbarkeit} % (fold)
\label{sub:Benutzbarkeit}

Die Software muss einfach verständlich sein ( ``Verständlichkeit'' ) und sich einfach Erlernen lassen (``Erlernbarkeit'').
Eine einfache Benutzung steht dabei im Vordergrund ( ``Benutzbarkeit'' ), hingegen soll ``Attraktivität'' nur soweit gewährleistet sein,
dass sie die ``Verständlichkeit'' positiv beeinflusst. 

% subsection Benutzbarkeit (end)


\subsection{Wartbarkeit} % (fold)
\label{sub:Wartbarkeit}

Es soll die Möglichkeit bestehen, leicht neue Funktionen zur Software hinzuzufügen ohne den konzeptionellen Aufbau des Produkts zu verändern. 
% subsection Wartbarkeit (end)

\subsection{Übertragbarkeit} % (fold)
\label{sub:Übertragbarkeit}

Auf die Qualitätsanforderung ``Übertragbarkeit'' soll in dem Sinne eingegangen werden, dass die Software auf Linux und Windows lauffähig ist ( ``Anpassbarkeit'' ).

% subsection Übertragbarkeit (end)
% section Nichtfunktionale Anforderungen (end)

% chapter Anforderungsdefinition (end)
 
\end{flushleft}

