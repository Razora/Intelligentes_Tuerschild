\chapter{ Zusammenfassung }
\vspace{-3cm}
\begin{flushleft}
Im Rahmen dieser Projektarbeit wurde mit Hilfe von Java, dem Betriebssystem ``Android'' und dafür entwickelte Software eine Möglichkeit entwickelt, welche ein Tablet-PC in einen digitalen Bilderrahmen verwandelt, um so den Aufwand der Pflege von Informationen zu verringern.
\newline

Im Einleitungskapitel Kapitel \ref{chap:Einleitung} wurde ein Überblick darüber gegeben, wie die aktuelle Situation in Bezug auf die Verteilung von Informationen an der Fachhochschule Stralsund aussieht und welche Veränderungen an diesem Zustand vorgenommen werden sollen. 
\newline

Daran anschließend wurde im Kapitel \ref{chap:Grundlagen} ``Grundlagen'' aufgezeigt, welche Technologien in der Arbeit genutzt werden, um die gestellte Aufgabe zu lösen. 
\newline

In Kapitel \ref{chap:Stand} ``Stand'' wurde noch einmal genauer beschrieben, wie die Türschilder bzw. Schaukästen genutzt werden und es wurde die Problemstellung noch einmal genauer definiert. Ebenfalls wurde eine bereits vorhandene Lösung für das Problem präsentiert und warum dieses nicht genutzt werden kann. 
\newline

Der nächste Abschnitt (Kapitel \ref{sec:Anforderungen}) gab eine Übersicht darüber, unter welchen Anforderungspunkten die Entwicklung des Projekts vollzogen werden sollte. Es erfolgte dafür eine Einteilung in funktionale und nichtfunktionale
Anforderungen. Dabei stützte sich die Erstellung der funktionalen Anforderungen auf die Formulierung von User Stories und die nichtfunktionalen Anforderungen auf den Standard ``ISO 9126''.
\newline

Das Kapitel \ref{chap:Entwurf} ``Entwurf'' setzte sich damit auseinander, welche Umsetzungsmöglichkeiten es für das vorliegende Problem gibt. Dabei entstanden zwei verschiedene Ansätze.
Als erstes wurde dabei Untersucht, ob ein digitaler Bilderrahmen als Lösung einsetzbar ist. Jedoch wurde während der Bearbeitung festgestellt, dass auf Grund einer neuer Firmwareversion des Bilderrahmens dieser nicht mehr für die Umsetzung geeignet ist. 
Es musste nun eine neue Option gesucht werden, um das Projekt weiterzuführen. Dabei fiel die Wahl auf einen Tablet-PC auf Android-Basis. Das Kapitel beschrieb daraufhin die Vorzüge des Tablet-PCs gegenüber dem Bilderrahmen und welche Arbeiten noch am Tablet durchgeführt werden müssen.
\newline

\newpage
Im Abschnitt \ref{chap:Umsetzung} wurde nun genauer darauf eingegangen, wie die im vorherigen Kapitel beschriebenen nötigen Arbeitsschritte umgesetzt wurden und wie die Versuche mit den digitalen Bilderrahmen aussahen.
\newline

%Das Kapitel \ref{chap:Leistungsbewertung} zeigte, welche zuvor aufgezeigten Probleme gelöst werden konnten und an welchen Stellen noch Verbesserungen vorgenommen werden können.

Das Kapitel \ref{chap:Ausblick} zeigte,was noch zu einer komfortablen endgültigen Lösung in der Fachhochschule umgesetzt werden müsste.
\end{flushleft}

% Zu einer runden Arbeit gehört auch eine Zusammenfassung, die eigenständig einen kurzen Abriß der Arbeit gibt. Eine halbe bis ganze DINA4 Seite ist angemessen. Dafür läßt sich keine Gebrauchsanweisung geben (für irgendetwas müssen die Betreuer ja auch noch da sein). Jetzt sollen noch einige, eher generelle Hinweise gegeben werden:

