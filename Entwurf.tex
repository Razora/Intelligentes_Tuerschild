\chapter{ Entwurf }
\vspace{-3cm}
\begin{flushleft}
Dieses Kapitel beschäftigt sich damit, welche Ziele die Arbeit verfolgt und wie diese mit den Anforderungen an das Endprodukt zusammengebracht werden. Weiterhin werden verschiedene Konzepte aufgezeigt, welche als Möglichkeit zur Problemlösung zur Verfügung standen. Diese werden daraufhin mit den Zielen und Anforderungen verbunden und es wird begründet, welche Lösung gewählt wurde.

\section{Ziele der Arbeit}\label{sec:Ziele der Arbeit}
Wie bereits beschrieben, verfolgt dieses Projekt verschiedene Intentionen. Diese werden an dieser Stelle noch einmal kurz aufgegriffen, um ein besseres Verständnis der folgenden Abschnitte zu gewährleisten.
\newline

Die Zielstellungen des Projekts lauten wie folgt:
\begin{itemize}
  \item Schaffung einer Lösung die bisherigen Türschilder zu ersetzen
  \item Verminderung des Verwaltungsaufwandes zur Eingabe und Pflege der zu publizierenden Informationen
\end{itemize}


\end{flushleft}

% st das zentrale Kapitel der Arbeit. Hier werden das Ziel sowie die eigenen Ideen, Wertungen, Entwurfsentscheidungen vorgebracht. Es kann sich lohnen, verschiedene Möglichkeiten durchzuspielen und dann explizit zu begründen, warum man sich für eine bestimmte entschieden hat. Dieses Kapitel sollte - zumindest in Stichworten - schon bei den ersten Festlegungen eines Entwurfs skizziert und in Stichworten geschrieben werden. Es wird sich aber in einer normal verlaufenden Arbeit dauernd etwas daran ändern. Das Kapitel darf nicht zu detailliert werden, sonst langweilt sich der Leser. Es ist sehr wichtig, das richtige Abstraktionsniveau zu finden. Bei der Verfassung sollte man auf die Wiederverwendbarkeit des Textes achten.
% 
% Plant man eine Veröffentlichung aus der Arbeit zu machen, können von diesem Kapitel Teile genommen werden. Das Kapitel wird in der Regel wohl mindestens 8 Seiten haben, mehr als 20 können ein Hinweis darauf sein, daß das Abstraktionsniveau verfehlt wurde.

