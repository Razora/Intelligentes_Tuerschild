\chapter{ Entwurf }
\vspace{-3cm}
\begin{flushleft}
Dieses Kapitel beschäftigt sich damit, welche Ziele die Arbeit verfolgt und wie diese mit den Anforderungen an das Endprodukt zusammengebracht werden. Weiterhin werden verschiedene Konzepte aufgezeigt, welche als Möglichkeit zur Problemlösung zur Verfügung standen. Diese werden daraufhin mit den Zielen und Anforderungen verbunden und es wird begründet, welche Lösung gewählt wurde.

\section{Ziele der Arbeit}\label{sec:Ziele der Arbeit}
Wie bereits beschrieben, verfolgt dieses Projekt verschiedene Intentionen. Diese werden an dieser Stelle noch einmal kurz aufgegriffen, um ein besseres Verständnis der folgenden Abschnitte zu gewährleisten.
\newline

Die Zielstellungen des Projekts lauten wie folgt:
\begin{itemize}
  \item Schaffung einer Lösung die bisherigen Türschilder zu ersetzen
  \item Verminderung des Verwaltungsaufwandes zur Eingabe und Pflege der zu publizierenden Informationen
\end{itemize}

Es muss allerdings berücksichtigt werden, dass diese Ziele nicht ohne Einschränkungen verfolgt werden können, denn sie sind an Anforderungen gebunden. Diese wurden bereits im Abschnitt \ref{sec:Anforderungen} ``Anforderungsdefinition'' vorgestellt.

\section{Konzepte zur Problemlösung}
Die in den Abschnitten \ref{sec:Ziele der Arbeit} und \ref{sec:Anforderungen} beschrieben Punkte müssen nun miteinander in Verbindung gebracht werden und es müssen, unter Berücksichtigung dieser Aspekte, mögliche Konzepte ausgearbeitet werden.
Diese Ausarbeitung und deren Ergebnisse werden in den folgenden Passagen genauer vorgestellt. Dabei werden zuerst verschiedene Lösungsansätze präsentiert, die dann bewertet werden. Im Anschluss daran wird einer der Vorschläge als Lösungsansatz gewählt und im nächsten Kapitel genauer betrachtet.

\subsection{Grundidee}
Für die Umsetzung des Projekts ist das bereits vorhandene Grundkonzept des energiesparenden PCs welcher mit einem Bildschirm in ein Gehäuse verbaut ist, die beste Grundlage. Jedoch ist es nicht möglich, die bereits fertigen Lösungen einzusetzen, da diese sehr kostenintensiv und unflexibel sind. Somit muss eine Abwandlung dieses Konzeptes entwickelt werden, um es für die Fachhochschule Stralsund einsatzfähig zu machen. 

\subsection{Nutzung eines digitalen Bilderrahmens als Türschild}
Die erste hier gezeigte Umsetzung ist es, die Grundidee mit Hilfe eines digitalen Bilderrahmens umzusetzen. Die Entscheidung dafür fiel unter anderem, weil zwei Geräte der Marke ``Kodak'' (Modell: W1030) bereits in der Fachhochschule vorhanden waren und es auch weitere Modelle gibt, die kostengünstig zu erhalten sind. Bei den zwei vorhandenen Bilderrahmen handelt es sich um Bilderrahmen mit der ``Pulse''-Funktion. 

Moderne digitale Bilderrahmen haben einen großen Funktionsumfang. Sie besitzen ,je nach Modell, neben der Fähigkeit der Darstellung von Bildern auch die Wiedergabe Videos und Audioaufnahmen. Um die Verwaltung der Geräte zu vereinfachen bauen die Hersteller eine WLAN-Schnittstelle in die Geräte ein. Somit bietet ein digitaler Bilderrahmen die Voraussetzungen, um auch als digitales Türschild eingesetzt zu werden. 

Wie in Kapitel \ref{sec:Anforderungen} ``Anforderungen'' beschrieben soll eine zentrale Möglichkeit geschaffen werden das Türschild zu verwalten. Da die bereitgestellten Bilderrahmen die in Kapitel \ref{chap:Grundlagen} ``Grundlagen'' erwähnte ``Pulse''-Funktion besitzen, bietet es sich an diese Funktion zur Umsetzung zu nutzen. Dafür muss jedoch erst einmal ein Konto bei Kodak angelegt werden. Dafür werden der vollständige Name und eine E-Mail-Adresse benötigt. Ist der Bilderrahmen mit einem WLAN-Netzwerk verbunden, kann nun im Kodak-Pulse-Service der Rahmen mit Hilfe einer Kennnummer (zu finden am Karton des Bilderrahmens oder bei der Ersteinrichtung der WLAN-Verbindung) das Gerät dem Konto hinzugefügt werden. Nun ist die Verwaltung über einen Webbrowser möglich. Es können nun nach belieben Fotos hinzugefügt und auch wieder gelöscht werden. Dies kann auch für mehrere Rahmen getan werden. Dafür müssen diese dem ``Pulse''-Konto einfach hinzugefügt werden. 
Jedoch ist diese Kombination von Hardware und Software nicht die erhoffte Lösung. Es besteht das Problem, dass mit der Nutzung des ``Kodak-Pulse''-Service die auf den Bilderrahmen gespeicherten Daten auf einen Kodak-Server übertragen werden. Es werden also somit Informationen die nur für die Fachhochschule und dessen Angehörige bestimmt sind, nach Außen getragen. Ein weiterer Kritikpunkt ist die Abhängigkeit von einer externen Firma. Sollten Fehler oder Softwareausfälle auftreten, können diese nicht in eigener Verantwortung behoben werden und es kann damit nicht mehr die volle Funktionalität der Türschilder gewährleistet werden. Weiterhin muss in die Überlegungen mit einbezogen werden, dass die Firma ``Kodak'' aktuelle in einer wirtschaftlichen Krise steckt. Nach der Bekanntgabe der Insolvenz von Kodak werden verschiedenste Patente und Services zum Kauf angeboten. Unter anderem auch der ``Pulse''-Service. Es muss also damit gerechnet werden, dass der ``Pulse''-Service aus Kostengründen eingestellt wird oder an eine andere Firma verkauft wird. Wird der Service dann eingestellt, wird der Bilderrahmen nicht mehr als Türschild nutzbar sein. 
Es wird also eine andere Lösung benötigt. Diese sollte vor allem die Option bieten alle nötigen Komponenten innerhalb der Fachhochschule zu halten.  

Um diese Forderung zu erfüllen, müssten vor allem die benötigten Server innerhalb der Fachhochschule stehen. Um dies zu ermöglichen, gibt es ein von Hajo Noerenberg geschriebenes PHP-Skript, welches als ``Kodak Pulse Emulator'' benannt ist. Grundidee dieses Emulators ist es alle Anfragen und Verbindungen der Bilderrahmen die eigentlich die Kodakserver erreichen sollen an dieses PHP-Skript umzuleiten. Dort werden dann Antwortpakete erstellt, die der Bilderrahmen so auffasst, dass dieser denkt er habe sich erfolgreich mit den Kodakservern verbunden. Mit dieser Technik besteht die Möglichkeit die Daten, die die Geräte halten, auf einem Server der Fachhochschule zu speichern. 
Jedoch stellt sich heraus, dass dieses Skript nicht mit den aktuellen Firmwareversionen der Bilderrahmen funktioniert. In den früheren Versionen wurde der Hostname des SSL-Zertifikates nicht überprüft. Jedoch wurde diese Prüfung mit den neuen Firmwareversionen eingeführt. Somit ist auch diese Methode nicht für die Problemlösung einsetzbar. 

Da jedoch der Kodak-Bilderrahmen keine weiteren Schnittstellen zur Verfügung stellt, kann dieser nicht für die Problemlösung genutzt werden. Es werden also andere Geräte mit weiteren Schnittstellen benötigt. Eine solche Schnittstelle wäre zum Beispiel eine ``Samba''-Schnittstelle. Es handelt dabei um ein Protokoll, welches in der Lage ist die Datei- und Druckdienste von Microsoft Windows zu emulieren. Mit diesem Protokoll können also Dateien auf einem Server freigegeben werden, die dann vom Bilderrahmen gelesen und auch wiedergegeben werden können. Die Suche nach Bilderrahmen die eine solche Möglichkeit bieten, ergab dass es einige Sony Modelle mit dieser Technik gibt. Jedoch sind diese sehr kostenintensiv in der Anschaffung. Außerdem muss wieder bedacht werden, dass die Rahmen ausschließlich für diesen Zweck genutzt werden können. 
So entstand die Idee nach einem anderen Gerät zu suchen, welches kostengünstiger ist und zudem noch einen größeren Funktionsumfang als die Bilderrahmen bietet. Die Wahl fiel daraufhin auf Android-Tablets kleinerer Hersteller. Diese sind schon für etwa 100 Euro zu finden und bieten bei der Entwicklung völlige Freiheit, da sie Eigenentwicklungen zulassen. 
Die Umsetzung mit Hilfe eines Tablets wird im nächsten Abschnitt genauer beschrieben. 

\subsection{Nutzung eines Android-Tablets als Türschild}
Wie breits im vorherigen Abschnitt erwähnt, muss eine Alternative gefunden werden, die weitere Schnittstellen zur Verfügung stellt. Das beste Preis-/Leistungsverhältnis dafür bietet ein Tablet-PC auf Android-Basis. Diese bieten die Möglichkeit bei entsprechender Hardware alle nötigen Änderungen vorzunehmen. Es stellt sich nun die Frage, welche Hardwarekomponenten in einem solchen Tablet verbaut sein müssen. Im speziellen Falle des Projektes ist es vollkommen ausreichend, wenn eine WLAN-Schnittstelle vorhanden ist. Alle weiteren nötigen Schritte können dann mit Hilfe von Software durchgeführt werden. 
Im nächsten Schritt muss nun entschieden werden, welche Funktionen das Tablet bereitstellen muss, um als Bilderrahmen eingesetzt werden zu können. Dafür muss jedoch festgelegt werden, welche Version des Android-Betriebssystems eingesetzt werden soll, denn diese ist entscheidend dafür, welche Software auf dem Gerät lauffähig ist und welche nicht und welches Tablet dann gewählt wird. Um eine die größtmögliche Aktualität der Software gewährleisten zu können, sollte das Tablet mindestens die Android-Version 4.0 unterstützen. Durch die Festlegung dieser Eigenschaft wird gleichzeitig die Anzahl der möglichen Geräte eingeschränkt. Somit muss das Tablet nun eine W-LAN-Schnittstelle bieten, mit Android 4.0 arbeiten und sollte in der Anschaffung um 100 Euro liegen. Nach einiger Recherche fiel die Entscheidung auf das ``Odys Xelio''. Dieses Android-Tablet erfüllt somit alle Voraussetzungen, um als Ersatz für den Bilderrahmen zu dienen. 
Nun müssen dem Tablet die nötigen Funktionen mit Hilfe von Software beigebracht werden. Es werden für den Einsatz folgende Funktionen benötigt:
\begin{itemize}
  \item das Tablet muss gegen unbefugte Eingaben von Dritten geschützt werden
  \item das Tablet muss in der Lage sein eine Fotopräsentation darzustellen
  \item das Tablet muss die Möglichkeit bieten, Bildmaterial von einer entfernten Quelle beziehen zu können
  \item das Tablet sollte so stromsparend wie möglich arbeiten
\end{itemize}

Diese Anforderungen müssen nun noch weiter präzisiert werden, denn nur so ist es möglich die richtigen Softwarekomponenten für den Produktiveinsatz auszuwählen. 
In Bezug auf den Punkt ``Verhindern von unbefugten Eingaben von Dritten'' bedeutet dies, dass jegliche Eingabemöglichkeit über das Touchdisplay verhindert werden muss, sobald eine Photopräsentation aktiviert ist. Da jedoch danach immer noch die Möglichkeit besteht mit Hilfe der Hardwaretasten des Odys Xelio zu navigieren, müssen diese ebenfalls deaktiviert werden. Somit ist gewährleistet, dass ein Außenstehender keine Mittel besitzt Einstellungen am Tablet zu verändern. 

Die Funktionalität der Darstellung einer Fotopräsentation ist deswegen notwendig, da die gezeigten Informationen als Bilddatei vorliegen. Somit muss das Tablet aus einer gegebenen Quelle die Dateien lesen und diese dann über den Bildschirm wiedergeben.

Die zuvor angesprochene Quelle besitzt dabei jedoch eine besondere Eigenschaft. Diese soll sich nämlich nicht direkt auf dem Tablet, sondern      
auf einem externen Medium befinden. Das Tablet selbst soll diese auslesen und dann, wenn es für die Ausführung nötig ist, zwischenspeichern und daraufhin wiedergeben.

Der letzte angesprochene Punkt ist der Energieverbrauch des Gerätes. Da die dargestellten Informationen nicht 24 Stunden pro Tag zugänglich sein müssen, soll eine Möglichkeit geschaffen werden, den Energieverbrauch in einem bestimmten Zeitrahmen zu senken. Ein Beispiel dafür wäre, dass der Bildschirm sich in den Stunden von 21 Uhr bis 7 Uhr ausschaltet oder dass das Gerät in diesen Stunden in einen energiesparenden Modus wechselt. 

All diese genannten Funktionen bietet das Odys Xelio jedoch nicht in seiner Standardkonfiguration. Es muss also noch mit Hilfe von weiteren Softwareprodukten diese Funktionalitäten geschaffen werden. Welche Software bzw. Techniken das im einzelnen ist, wird im nächsten Abschnitt genauer beschrieben. 
%danach anforderungen spezifizieren
%erklären, dass das tablet das so nicht kann
%hinweis darauf, dass im nächsten abschnitt darauf eingegangen wird
\end{flushleft}

% st das zentrale Kapitel der Arbeit. Hier werden das Ziel sowie die eigenen Ideen, Wertungen, Entwurfsentscheidungen vorgebracht. Es kann sich lohnen, verschiedene Möglichkeiten durchzuspielen und dann explizit zu begründen, warum man sich für eine bestimmte entschieden hat. Dieses Kapitel sollte - zumindest in Stichworten - schon bei den ersten Festlegungen eines Entwurfs skizziert und in Stichworten geschrieben werden. Es wird sich aber in einer normal verlaufenden Arbeit dauernd etwas daran ändern. Das Kapitel darf nicht zu detailliert werden, sonst langweilt sich der Leser. Es ist sehr wichtig, das richtige Abstraktionsniveau zu finden. Bei der Verfassung sollte man auf die Wiederverwendbarkeit des Textes achten.
% 
% Plant man eine Veröffentlichung aus der Arbeit zu machen, können von diesem Kapitel Teile genommen werden. Das Kapitel wird in der Regel wohl mindestens 8 Seiten haben, mehr als 20 können ein Hinweis darauf sein, daß das Abstraktionsniveau verfehlt wurde.

