\chapter{ Grundlagen }
\vspace*{-3cm}
\begin{flushleft}

In diesem Abschnitt der Arbeit, werden verschiedene Begriffe genauer erläutert, die benötigt werden, um die später beschrieben Bearbeitung
des Themas nachvollziehen zu können. Dazu gehören zum einen die Definitionen der eingesetzten Geräte und die dazugehörige Software.

\section{Digitaler Bilderrahmen}
Ausgangspunkt der Arbeiten am ``intelligenten Türschild'' war ein sogenannter digitaler Bilderrahmen der Firma Kodak. Da jedoch diese digitalen Bilderrahmen oft verschiedene Ausstattungen besitzen und sich somit gravierend unterscheiden können, wird zunächst eine Definition benötigt.

Bei einem digitalen Bilderrahmen handelt es sich um ein elektronisches Gerät, welches als Haupteinsatzgebiet die Wiedergabe von Bildmaterial hat. Es können dann noch je nach Modell weitere Funktionen (z.B. die Wiedergabe von Ton- und Videomaterial) und Ausstattungsmerkmale (z.B. Lautsprecher oder Kartenlesegeräte) vorhanden sein. Jedoch handelt es sich bei diesen Geräten nur um eine meist vereinfachte Form der derzeitigen Bildschirmtechnologie (z.B. LCD-Bildschirme, welche die Anzeige mit Hilfe von Flüssigkristallen erzeugen), welche dann mit einem Rahmen aus Kunststoff oder Metall ausgestattet werden.

Neben den unterschiedlichen Hardwareausstattungen verwenden die Hersteller verschiedene Softwarelösungen, um den Nutzer die Möglichkeit zu geben dem Bilderrahmen neues Bildmaterial zur Verfügung zu stellen oder um Bilder mit anderen Personen zu teilen. Die eingesetzte Software unterscheidet sich jedoch von Hersteller zu Hersteller. So setzt zum Beispiel Kodak als Software ``Kodak Pulse'' ein. Diese wird nun im nächsten Absatz genauer dargestellt, da sie für die weitere Bearbeitung des Themas nötig ist.

\section{Kodak Pulse}
Bei der Software die auf den Namen ``Kodak Pulse'' hört, handelt es sich um einen Service den die Firma Kodak ihren Kunden zur Verfügung stellt, um von Kodak produzierte digitale Bilderrahmen zu verwalten. Dazu muss jedoch der Bilderrahmen mit einem eingebauten W-Lan-Modul ausgerüstet sein. Der Nutzer muss nun wie in der Gebrauchsanweisung beschrieben den Bilderrahmen mit dem Internet verbinden. Im nächsten Schritt muss eine Anmeldung auf der Internetpräsenz von Kodak vorgenommen werden. Im Verlaufe dieser Anmeldung wird auch der Bilderrahmen mit dem Online-Konto verknüpft. Es besteht nun die Möglichkeit über die Weboberfläche Bilder zum Speicher des Rahmens hinzuzufügen oder zu löschen. Weiterhin kann der Nutzer die im Bilderrahmen vorhandenen Bilder mit Hilfe verschiedener Internetdienste wie Facebook oder Twitter mit anderen Personen teilen.

Jedoch ist die Fortführung dieses Dienstes gefährdet. Da in jüngster Vergangenheit immer mehr finanzielle Probleme bei Kodak auftraten, ist es nun schon zum Verkauf einiger Technologien und Diensten gekommen. Auch der ``Kodak Pulse''-Dienst steht zum Verkauf. Jedoch ist noch nicht sicher ob der Dienst im Falle eines Verkaufs noch weitergeführt wird. Dieser Fakt muss in diesem Abschnitt erwähnt werden, da dies den Verlauf der Bearbeitung entscheidend beeinflusst hat. 

\section{Android}
Bei dem Softwareprodukt ``Android'' handelt es sich um ein Betriebssystem, welches vorrangig auf mobilen Geräten wie Handys bzw. Smartphones und Tablets zum Einsatz kommt. Entwickelt wird diese Software von der ``Open Handset Alliance''. Dies ist ein Zusammenschluss mehrerer Unternehmen, welche das Ziel haben offene Standards für mobile Geräte zu schaffen. Bekannte Unternehmen die diesem Verbund angehören sind Google, Ebay, Vodafone und Gerätehersteller wie HTC, LG Electronics oder Samsung Electronics. 
Als Grundlage für das Android-Betriebssystem ist der Linux-Kernel (die unterste Softwareschicht eines Betriebssystems) in der Version 2.6 herangezogen worden. Seit Android-Version 4 ist auch ein Linux-Kernel der Version 3.x möglich. 
Das Endgerät, auf dem das Android-Betriebssystem arbeitet, kann mit Hilfe sogenannter ``Apps'' (Software die auf einem Android-Gerät nachträglich installiert werden kann) welche von verschiedenen Anbietern heruntergeladen werden können,um weitere Funktionen erweitert werden. Derzeit gibt es etwas mehr als 500.000 dieser Apps (Stand August 2012). Somit kann ein Endgerät je nach eingesetzten Programmen verschiedene Aufgaben erfüllen. 

\section{ADB}
Im späteren Verlauf der Projektarbeit kommt die sogenannte ``ADB'' zum Einsatz. Diese Abkürzung steht für ``Android Debug Bridge''. Hinter diesem Begriff verbirgt sich ein Tool, welches die direkte Kommunikation mit dem Gerät ermöglicht. Es wird ausschließlich über die Kommandozeile des PCs bedient und stellt verschiedene Befehle zur Verfügung. So können zum Beispiel Dateien vom PC auf das Handy kopiert werden oder auch vom Handy gelesen werden. Weiterhin besteht auch die Möglichkeit weitere Programme mit Hilfe dieses Tools auf dem Endgerät zu installieren. Damit bietet die Android Debug Bridge die Möglichkeit, am Gerät schwer zu findende Einstellungen vom PC vorzunehmen und diese Vorgänge zu vereinfachen. 

\section{Rooten von Android} 
Unter dem Begriff ``Root'' kann man den Administrator eines Gerätes oder eines Systems verstehen. Dieser ist dabei mit den größtmöglichen Rechten innerhalb des Systems ausgestattet. Standardmäßig besteht bei Android eine Einschränkung der Rechte. Somit muss ein sogenanntes ``Rooting'' durchgeführt werden, um diese Einschränkung zu umgehen. Die Prozedur für das Rooting unterscheidet sich von Gerät zu Gerät. Hat man allerdings nun vollen Zugriff auf das Gerät können nun Einstellungen durchgeführt werden, die vorher nicht möglich waren, zum Beispiel das Installieren eines veränderten Android-Betriebssystems (sogenannte ``Custom-Roms'') oder das nutzen von Apps die tief in das System eingreifen. 
\end{flushleft}

% Hier werden zwei wesentliche Aufgaben erledigt:
% 
% Der Leser muß alles beigebracht bekommen, was er zum Verständnis der späteren Kapitel braucht. Insbesondere sind in unserem Fach die Systemvoraussetzungen zu klären, die man später benutzt. Zulässig ist auch, daß man hier auf Tutorials oder Ähnliches verweist, die hier auf dem Netz zugänglich sind.
% Es muß klar werden, was anderswo zu diesem Problem gearbeitet wird. Insbesondere sollen natürlich die Lücken der anderen klar werden. Warum ist die eigene Arbeit, der eigene Ansatz wichtig, um hier den Stand der Technik weiterzubringen? Dieses Kapitel wird von vielen Lesern übergangen (nicht aber vom Gutachter ;-), auch später bei Veröffentlichungen ist "Related Work" eine wichtige Sache.
% Viele Leser stellen dann später fest, daß sie einige der Grundlagen doch brauchen und blättern zurück. Deshalb ist es gut, Rückwärtsverweise in späteren Kapiteln zu haben, und zwar so, daß man die Abschnitte, auf die verwiesen wird, auch für sich lesen kann. Diese Kapitel kann relativ lang werden, je größer der Kontext der Arbeit, desto länger. Es lohnt sich auch! Den Text kann man unter Umständen wiederverwenden, indem man ihn als "Tutorial" zu einem Gebiet auch dem Netz zugänglich macht.
% 
% Dadurch gewinnt man manchmal wertvolle Hinweise von Kollegen. Dieses Kapitel wird in der Regel zuerst geschrieben und ist das Einfachste (oder das Schwerste weil erste).

