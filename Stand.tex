% Wie wird aktuell mit Türschildern gearbeitet
\chapter{ Stand }\label{chap:Stand}
\vspace{-3cm}
\begin{flushleft}
Dieses Kapitel setzt sich damit auseinander, wie aktuell die herkömmlichen Türschilder eingesetzt werden und welche möglichen Lösungen existieren, um diese zu ersetzen.

\section{Nutzung der Türschilder innerhalb der Gebäude der Fachhochschule} 
Zum aktuellen Zeitpunkt werden vor den Seminarräumen und den Laboren Metallschilder und vereinzelt Schaukästen bzw. Pinnwände eingesetzt, um Informationen und Stundenpläne zu veröffentlichen. Dabei ist für jeden Raum eine separate Betreuung nötig. Dies bedeutet, dass für jede neue Information die vor dem Raum publiziert werden soll, ein neues Dokument erstellt werden und platziert werden muss. Da wie bereits beschrieben nur ein sehr eingeschränkter Raum zur Verfügung steht, muss ebenfalls entschieden werden, welche Informationen nicht mehr benötigt werden und ausgetauscht werden können. Somit ist für einen eigentlich einfachen Vorgang ein dafür großer Arbeitsaufwand nötig. Ein weiterer Nachteil der bisherigen Lösung ist, dass durch die dezentrale Verwaltung es dazu kommen kann, dass ein und dieselbe Information mehrfach ausgehängt wird. Es kann also dazu kommen, dass an einer Stelle die Information bereits nicht mehr als aktuell angesehen wird, während es an anderer Stelle immer noch für aktuell gehalten wird. Es muss somit eine Lösung geschaffen werden, die eine zentrale Verwaltung ermöglicht um damit den Arbeitsaufwand zu verringern und gleichzeitig die Aktualität von Informationen zu verbessern. 
Für dieses Problem gibt es einen Lösungsansätze, welcher im nächsten Abschnitt genauer betrachtet wird. 

\section{Welche Technik gibt es bereits?}
\end{flushleft}


